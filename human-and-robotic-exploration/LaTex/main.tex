\documentclass[12pt,a4paper,twoside,openrigth]{book}

\usepackage[ruled]{algorithm2e}
\usepackage{amsmath}
\usepackage{amssymb}
\usepackage{booktabs} 
\usepackage{caption}
\usepackage{emptypage}
\usepackage{fancyhdr}
\usepackage[hang,flushmargin]{footmisc}
% \usepackage{hyperref}
\usepackage{graphicx}
\usepackage[utf8]{inputenc}
\usepackage{multirow}
\usepackage{pgfplotstable}
\usepackage{ragged2e}
\usepackage{setspace}
\usepackage{siunitx}
\usepackage{subcaption}
\usepackage{tabularx}
\usepackage{titlepic}
\usepackage{url} 
\usepackage[nottoc]{tocbibind}

% GENERAL SETUP %

% Customize header and footer.
\pagestyle{fancy}
\fancyhf{}
\fancyhead[LE]{\slshape\nouppercase{\textnormal{\leftmark}}}
\fancyhead[RO]{\slshape\nouppercase{\textnormal{\rightmark}}}
\fancyfoot{}
\fancyfoot[RO, LE]{\thepage}

% Force footnotes to stay on the same page they are defined.
\interfootnotelinepenalty=10000

% Limit decimal to 2 digits.
\sisetup{
  round-mode = places,
  round-precision = 2,
}

% Limit the TOC to the subsubsections.
\setcounter{tocdepth}{3}

% Number the subsubsections.
\setcounter{secnumdepth}{3} 
    
% Set the image folder path.
\graphicspath{ {images/} }

% Relax italics in the TOC.
\let\LaTeXStandardTableOfContents\tableofcontents
\renewcommand{\tableofcontents}{\begingroup\renewcommand{\itshape}{\relax}\LaTeXStandardTableOfContents\endgroup}

% Define footnote for algorithms.
\makeatletter
\newcommand{\algorithmfootnote}[2][\footnotesize]{
	\let\old@algocf@finish\@algocf@finish
	\def\@algocf@finish{\old@algocf@finish
	\leavevmode\rlap{\begin{minipage}{\linewidth}
	#1#2\end{minipage}}}
}

% Add ''and'' and ''or'' keywords to the algorithm environment.
\SetKw{And}{and}
\SetKw{Or}{or}

% Define centering column with automatic newline in table.
\newcolumntype{C}{>{\Centering\arraybackslash}X}

% PDF SETUP %

% Set the left and the right margin to the same width.
\setlength\oddsidemargin{\dimexpr(\paperwidth-\textwidth)/2 - 1in\relax}
\setlength\evensidemargin{\oddsidemargin}

% MACROS %

% Macro for italics.
\def\<#1>{\textit{#1}}
% Macro for bold (changed to italics to uniform).
\def\[#1]{\textit{#1}}

% MATHEMATICAL OPERATORS %

\DeclareMathOperator*{\argmin}{arg\,min}
\DeclareMathOperator*{\argmax}{arg\,max}
\DeclareMathOperator*{\degcent}{deg}
\DeclareMathOperator*{\spl}{d_{sp}}
\DeclareMathOperator*{\diam}{diam}
\DeclareMathOperator*{\neigh}{neighborhood}
\DeclareMathOperator*{\walld}{d_{wall}}
\DeclareMathOperator*{\maxwalld}{maximum\,wall\,distance}
\DeclareMathOperator*{\cartd}{d}
\DeclareMathOperator*{\mapd}{map\,diagonal}
\DeclareMathOperator*{\intd}{d_{int}}
\DeclareMathOperator*{\mode}{Mo}

% Do something about argmin and argmax.
\newcommand{\cs}[1]{\texttt{\symbol{`\\}#1}}

% Define new sizes for the parenthesis.
\makeatletter
\newcommand{\biggg}{\bBigg@{2.25}}
\newcommand{\Biggg}{\bBigg@{2.5}}
\newcommand{\vast}{\bBigg@{3.5}}
\newcommand{\Vast}{\bBigg@{4}}
\makeatother

% FRONT MATTER %

\begin{document}

\frontmatter

% TITLE PAGE

\begin{titlepage}
\centering

{\scshape\LARGE Politecnico di Milano \par}
{\Large Corso di Laurea Magistrale in Ingegneria Informatica \par}
{\Large Dipartimento di Elettronica, Informazione e Bioingegneria}

\vspace{1cm} \includegraphics[width=0.3\textwidth]{polimi}\par \vspace{1cm}

{\scshape\LARGE An Online Framework for User-Based Analysis of Maps in First Person Shooters \par}

\raggedright
\vfill\Large Supervisor: Professor Daniele LOIACONO

\vfill \begin{minipage}[t]{0.4\textwidth}\end{minipage} \hfill
\begin{minipage}[t]{0.6\textwidth} \begin{flushright}
\raggedright Final thesis by: \par Marco BALLABIO Matr. 857169
\end{flushright} \end{minipage}

\centering
\vfill\Large Academic year 2016/2017

\end{titlepage}

% ABSTRACTS AND THANKS %

\chapter{\textit{Thanks}}

\indent 

\textit{Firstly, I would like to thank Professor Daniele Loiacono for his precious help during the months that led to the completion of this work.}

\par \mbox{}

\textit{I would also like to thank my parents, who gave me the opportunity to focus on my studies, my grandmother, who constantly supported me, and my aunt, uncle and cousins from Milan, who often hosted me after a long day of lectures.}

\par \mbox{}

\textit{Finally, I would like to thank my colleague and friend Luca, essential companion in this journey, and all the other friends with whom I shared these intense years.}

\par \mbox{}

\textit{\rightline{Marco Ballabio}}

\chapter{Abstract}

\<Level design> plays a key role in the development of a video game, since it allows to transform the \<game design> in the actual \<gameplay> that the final user is going to experience. Nevertheless, we are still far from a scientific approach to the subject, with a complete lack of a shared terminology and almost no experimental validation for the most used techniques. Even if the video game industry does not acknowledge this problem, in the last years the academic environments have shown an increasing interest towards this subject. \\
We analyzed the main breakthroughs made in level design research applied to the genre of \<First Person Shooters>, devoting particular attention to the ones that try to assist the design process by employing \<Procedural Content Generation>. To support this kind of research, we developed an \<open-source> \<framework> that employs procedural algorithms to generate maps with different topologies, both \<single-level> and \<multi-level>, but that also allows to import maps generated in previous works, thanks to a broad support to the most common export formats used in the literature. The framework was also designed for providing an easy way to define and deploy \<browser-playable online experiments>, that allow to analyze how real users react to different contents. \\
We also explored a novel approach for the analysis of First Person Shooter levels, that uses \<Graph Theory> to extract information about the layout of a map. We used this information to define an approach that uses \<heuristics> to place game elements considering the layout of the map and the features of each element.

\chapter{Sintesi}

Il \<level design> gioca un ruolo chiave nello sviluppo di un videogioco, dal momento che permette di trasformare il \<game design> nell'effettiva esperienza di \<gameplay> che verrà sperimentata dall'utente finale. Nonostante ciò, siamo ancora lontani da un approccio scientifico verso la materia, a causa della completa mancanza di un vocabolario condiviso e della quasi totale assenza di validazione sperimentale per le tecniche più comuni. Anche se l'industria tende ad ignorare questo problema, negli ultimi anni gli ambienti accademici hanno mostrato un crescente interesse verso questo campo. \\
Abbiamo analizzato le principali scoperte fatte nel campo del \<level design> applicato al genere dei \<First Person Shooter>, riservando particolare attenzione ai casi in cui si usa la \<Generazione Procedurali di Contenuti> per assistere il processo di design. Per agevolare questo tipo di ricerca, abbiamo sviluppato un \<framework> \<open-source> che si avvale di algoritmi procedurali per generare mappe con topologie differenti, con uno o più piani, ma che permette anche di importare le mappe generate nei lavori precedenti, grazie ad un vasto supporto per i formati di esportazione più diffusi in questo campo. Il framework è stato anche progettato per consentire la facile creazione di \<esperimenti online giocabili da browser>, che permettono di analizzare come degli utenti reali reagiscono a differenti tipi di contenuto. \\
Abbiamo anche esplorato un nuovo approccio per l'analisi dei livelli per First Person Shooter, che si avvale della \<Teoria dei Grafi> per estrarre informazioni riguardanti il \<layout> di una mappa. Utilizzando queste informazioni, abbiamo definito un approccio basato su \<euristiche> per disporre gli elementi di gioco tenendo conto del layout della mappa e delle caratteristiche di ciascun elemento.


\tableofcontents

\listoffigures

\listoftables

% MAIN MATTER %

\mainmatter

% CHAPTER 1 - INTRODUCTION AND MOTIVATION %

\chapter{Introduction}

Video games are really complex products, but it is rather simple to point out three elements that mainly influence their commercial success: \<visuals>, \[gameplay]\footnote{The specific way in which players interact with the game.} and \<narrative>. If in the past visual improvements were impressively fast, with games looking more and more realistic from year to year, lately this progress consistently slowed down, leaving narrative and gameplay as the main selling points. Since video games are interactive products, the latter is the one that influences user experience the most. Gameplay is defined by a set of rules commonly referred as \<game design>. To the player, game design is presented in a tangible way via \<level design>, which consists in the creation of the worlds where the game takes place. This is a critical component, since an inadequate level design can easily compromise the whole experience.

\par

One of the most successful video game genre is the one of \[First Person Shooters], that thanks to its first player perspective allows the user to experiment a complete immersion in the game world. From the very beginning, this has required a close attention to level design, that underwent a constant evolution, up to the stable situation of the recent years. Furthermore, the ever-increasing success of competitive multiplayer added a new level of complexity to the creation of maps, that need to support different game modes, play styles and interactions, allowing a fun and challenging gameplay to arise naturally.

\section{Motivations and purpose}

Despite the importance of level design for the FPS genre, the video game industry has never attempted a scientific approach to this field. Consequently, game design is a rather abstract discipline, with no common vocabulary or well-defined standards, but rather based on the experience of who is working in this field from many years. This affects also the related literature, that is confined to listing the most used patterns and conventions, without focusing on why they work.

\par

In the last years, instead, academic environments started to address this discipline with increasing interest. The researches performed in this field revolve around the identification and definition of design patterns and design techniques, with a deep analysis of how and why they work, and the creation of novel approaches to level design. Some of these techniques try to automate the design process by employing \<procedural generation>, often combined with \<evolutionary algorithms>. 

\par

A problem peculiar to this kind of research is how difficult it is to obtain information from real users, since this requires them to either download a specific game or to take part in play-test sessions and both option discourage participation.

\par

The aim of this thesis is to solve this problem and to attempt a novel approach to map analysis and game element placement. The former was addressed by designing an open-source \<framework> capable of generating and importing single-level and multi-level maps and of deploying online experiments to collect data from real users via a browser-playable FPS game. The latter, instead, was achieved by using \<Graph Theory> to compute topological metrics from multiple graph representation of the maps, each one highlighting different features, and by using the information provided by these metrics to strategically place game elements, paying attention to the gameplay dynamics associated with each one of them.

\section{Synopsis}

The contents of the thesis are the followings:

\par \mbox{}

In the second chapter we describe the state of the art of First Person Shooters, both in academic research and in commercial games, paying attention on how level design practices and procedural content generation are applied in this genre. We also list some examples of how Graph Theory is employed in video games.

\par
\mbox{}

In the third chapter we present the framework that we have developed, analyzing its features and its overall structure.

\par
\mbox{}

In the fourth chapter we present an approach that uses Graph Theory to place game elements in procedurally generated levels, with an overview of the theory and the assumptions behind it.

\par
\mbox{}

In the fifth chapter we describe an experiment performed with our framework for validating the heuristics that we have defined for the placement of spawn points.

\par 
\mbox{}

In the concluding chapter we evaluate the obtained results and we analyze the potential future developments of this work.

% CHAPTER 2 - STATE OF THE ART %

\chapter{State of the art}

% INTRODUCTION %

In this chapter we analyze the current state of \<Level Design> and of its common practices, both in academic and in professional environments, with attention to the genre of \<First Person Shooters> (or \<FPS>).

\par

We then talk about \<Procedural Content Generation> (or \<PCG>), focusing on how it allows to enrich and ease the design process.

\par

After that, we give an overview of the First Person Shooter genre, analyzing its features, history and evolution, devoting special attention to the games that mostly contributed to the definition of this genre.

\par

Finally, we analyze how Graph Theory has been in used in Video Games during the years.

% LEVEL DESIGN %

\section{Level Design Theory}

\<Level Design> is a game development discipline focused on the creation of video game levels.

\par

Today, the level designer is a well-defined and fundamental figure in the development of a game, but it was not always so. In the early days of the video game industry, it was a widespread practice to assign the development of levels to members of the team with other roles, usually programmers. Apart from the limited number of team members and the low budget, this was because there were no tools such as \<level editors>\footnote{\label{levelEditorFootnote}A level editor is a software used to design levels, maps and virtual worlds for a video game.}, that allowed the level designer to work on a level without being involved with code.

\par

The level designer has a really significant role in the development of a good game, since he is responsible for the creation of the world and for how the player interacts with it. The level designer takes an idea and makes it tangible. Despite the importance of this process, after all this years it has not been established a common ground or a set of standards and level design is still considered as a form of art, based on heuristics, observation, previous solutions and personal sensibility.

\par

In addition to gameplay, the game designer must consider the visual appearance of the level and the technological limitations of the \<game engine>\footnote{\label{gameEngineFootnote}A game engine is a software framework designed for the creation and development of video games.}, combining all this elements in a harmonic way.

\par

One of the core components of level design is the \[``level flow'']. For single player games it translates into the series of actions and movements that the player needs to perform to complete a level. A proficient level design practice is to guide the player in a transparent way, by directing his attention towards the path he needs to follow. This can be achieved in diverse ways. Power ups and items can be used as breadcrumbs to suggest the right direction in a one-way fashion, since they disappear once picked up. Lighting, illumination and distinctly colored objects are another common approach to this problem. A brilliant example of this is \<Mirror's Edge>\footnote{Digital Illusions CE, 2008.}, which uses a really clear color code, with red interactive objects in an otherwise white world, to guide the player through its fast-paced levels. There are also even more inventive solutions, like the dynamic flock of birds in \<Half Life 2>\footnote{Valve, 2004.}, used to catch the player attention or to warn him of incoming dangers\cite{GuidingThePlayersEye}. Finally, sounds and architectures are other elements that can be used to guide the player. In the academic environment, a lot of researchers have analyzed the effectiveness of this kind of solutions: Alotto\cite{HowLevelDesignersAffect} considers how architecture influences the decisions of the player, whereas Hoeg\cite{TheInvisibleHand} also considers the effect of sounds, objects and illumination, with the last being the focus of Brownmiller's\cite{InGameLigthing} work.

\par

In multiplayer games the level flow is defined by how the players interact with each other and with the environment. Because of this, the control of the level designer is less direct and is exercised almost exclusively by modeling the map. Considering FPS, the level flow changes depending on how much an area is attractive for a player. The more an area is easy to navigate or offers tactical advantage, such as covers, resources or high ground, the more players will be comfortable moving in it. This doesn't mean that all areas need to be designed like this, since zones with a ``bad'' flow but an attractive reward, such as a powerful weapon, force the player to evaluate risks and benefits, making the gameplay more engaging. The conformation of the map and the positioning of interesting resources are used to obtain what Güttler et al.\cite{Guttler:2003:SPL:963900.963915} define as ``points of collisions'', i.e. zones of the map were the majority of the fights are bound to happen. 

\par

Moving back to academic research, Güttler et al. have also noticed how aesthetic design loses importance in a multiplayer context. Other researches are instead focused on finding \[patterns] in the design of multiplayer maps: Larsen\cite{LevelDesignPatterns} analyzes three really different multiplayer games, \<Unreal Tournament 2004>\footnote{Epic Games, 2004.}, \<Day of Defeat: Source>\footnote{Valve, 2005.} and \<Battlefield 1942>\footnote{DICE, 2002.}, identifying shared patterns and measuring their effect on gameplay, suggesting some guidelines on how to use them, whereas Hullet and Whitehead identify some patterns for single player games\cite{Hullett:2010:DPF:1822348.1822359}, many of whom are compatible with a multiplayer setting, with Hullett also proving cause-effect relationships for some of this patterns by confronting hypothesized results with the ones observed on a sample of real players\cite{TheScienceOfLevelDesign}. Despite these experimental results contributing to a formalization of level design, we are still far from a structured scientific approach to the subject.

% PROCEDURAL CONTENT GENERATION %

\section{Procedural Content Generation}

\<Procedural Content Generation> refers to a family of algorithms used to create data and content in an automatic fashion. In game development it is commonly used to generate weapons, objects, maps and levels, but it is also employed for producing textures, models, animations, music and dialogues.

\par

The first popular game to use this technique was \<Rogue>\footnote{Michael Toy, Glenn Wichman, 1980.}, an ASCII dungeon exploration game released in 1980, where the rooms, hallways, monsters, and treasures the player was going to find were generated in a pseudo-random fashion at each playthrough. Besides providing a huge replay value to a game, PCG allowed to overcome the strict memory limitations of the early computers. Many games used pseudo-random generators with predefined \<seed values> to create very large game worlds that appeared to be premade. For instance, the space exploration and trading game \<Elite>\footnote{David Braben, Ian Bell, 1984.} contained only eight galaxies, each one with 256 solar systems, of the possible 282 trillion the code was able to generate, since the publisher was afraid that such an high number could cause disbelief in the players. Another example is the open world action role-playing game \<The Elder Scrolls II: Daggerfall>\footnote{Bethesda Softworks, 1996.}, which game world has the same size of Great Britain. 

\par

As computer hardware advanced and the size of the memory increased, procedural generation of game worlds was generally put aside, since it could not compete with the level of detail that hand-crafted worlds were able to achieve.

\par

However, in the last years, with the players' expectations and the production value of video games constantly increasing, procedural generation made a comeback as a way to automate the development process and reduce costs. Many \<middleware> tools, such as \<SpeedTree>\footnote{IDV, Inc.} and \<World Machine>\footnote{World Machine Software, LLC.}, are used to produce various kind of content, like terrain and natural or artificial environments.

\par

Many modern \<AAA>\footnote{Video games produced and distributed by a major publisher, typically having high development and marketing budgets.} games use procedural generation: in \<Borderlands>\footnote{Gearbox Software, 2009.} a procedural algorithm is responsible for the generation of guns and other pieces of equipment, with over a million unique combinations; in \<Left 4 Dead>\footnote{Valve, 2008.} an artificial intelligence is used to constantly make the players feel under threat, by dynamically changing the music, spawning waves of enemies and changing the accessible paths of the level; in \<Spore>\footnote{Maxis, 2008.} \<procedural animation> is employed to determine how the creatures created by the player move.

\par

Nowadays, PCG is widely used by \<independent> developers, that, lacking the high budgets of AAA games, try to obtain engaging and unusual gameplay using unconventional means. The most famous example is \<Minecraft>\footnote{\label{ }Mojang, 2011.}, a sandbox survival game which worlds, composed exclusively by cubes, are generated automatically. Currently, the most extreme form of procedural generation is the one found in \<No Man's Sky>\footnote{Hello Games, 2016.}, a space exploration game where space stations, star-ships, planets, trees, resources, buildings, animals, weapons and even missions are generated procedurally. Following in the footstep of their forefather, many roguelike games still use PCG, like \<The Binding of Isaac>\footnote{Edmund McMillen, 2011.}.

\par

All the algorithms used by these games and middleware are designed to be as fast as possible, since they need to generate the content in real time. In the last years researchers have nevertheless tried to explore new paradigms, creating more complex procedural generation techniques, that allow for a tighter control on the output. Being one of the problems of PCG the lack of an assured minimum quality on the produced content, the academic environment has focused not only on more advanced generation algorithms, but also on techniques to evaluate the output itself in an \<automatic> fashion. In this field, Togelius et al.\cite{10.1007/978-3-642-12239-2_15} defined \<Search-Based Procedural Content Generation>, a particular kind of \<Generate-And-Test>\footnote{Algorithms with both a generation and an evaluation component, that depending on some criterion, decide to keep the current result or to generate a new one.} algorithm, where the generated content, instead of being just accepted or discarded, is evaluated assigning a \<suitability score> obtained from a \<fitness function>, used to select the best candidates for the next iterations.

% PROCEDURAL CONTENT GENERATION IN FPSES %

\section{Procedural Content Generation for FPS maps}

We have really few examples of commercial FPS that use PCG to generate their maps: with the exception of \<Soldier of Fortune II: Double Helix>\footnote{ Raven Software, 2002.}, that employs these techniques to generate whole missions, the few other cases we have are all roguelikes with a FPS gameplay, like \<STRAFE>\footnote{Pixel Titans, 2017.}.

\par

Despite the total lack of FPS using procedural generation to obtain multiplayer maps, researchers have proved that search-based procedural content generation can be an useful tool in this field. In a seminal work, Cardamone et al.\cite{Cardamone:2011:EIM:2008402.2008411} tried to understand which kind of \<deathmatch>\footnote{A widely used multiplayer game mode where the goal of each player is to kill as many other players as possible until a certain end condition is reached, commonly being a kill limit or a time limit.} maps created the most enjoyable gameplay possible. To achieve this, the authors generated maps for \<Cube 2: Sauerbraten>\footnote{Wouter van Oortmerssen, 2004} by maximizing a fitness function computed on \<fight time> data collected from \<simulations>\footnote{In the field of search-based procedural generation, fitness function based on simulation are computed on the data collected from a match between artificial agents in the map at issue. They differ from \<direct> and \<interactive> functions, that evaluate, respectively, the generated content and the interaction with a real player.}, with the fight time being the time between the start of a fight and the death of one of the two contenders. The choice of this fitness function is based on the consideration that a long fight is correlated with the presence of interesting features in the map, such as escape or flanking routes, hideouts and well positioned resources. 

\par

Stucchi et al.\cite{EvoluzioneMappeBilanciate}, yet remaining in the same field, attempted a completely different use of procedural generation, by producing balanced maps for player with different weapons or different levels of skill. For doing so, they generated procedural maps via evolutionary algorithms, evaluating them with a fitness function based on simulation that computes the entropy of kills. Starting from a situation where one of the two players has a significant advantage, they proved that changes in the map structure allow to achieve a significant balance increase.

\par

Arnaboldi\cite{SviluppoDiUnFramework} combined these two approaches, creating a framework that automatically produces maps using a genetic process like the one of Cardamone. In Arnaboldi's work, however, the fitness function is way more complex, since it considers a high number of gameplay metrics, and the AI of the employed \<bots>\footnote{The artificial players of a video game.} is closer to the one of a human player, thanks to a series of adjustments made to the stock \<Cube 2> one. These improvements significantly increase the flexibility of the framework and the overall quality of the output, allowing to identify and analyze some recurring patterns and their relationship with the statistics gathered during the simulations. 

\par

Ølsted et al.\cite{DesignerJob} moved the focus of their research from deathmacth to squad game modes with specific objectives, sustaining not only that the maps generated by Cardamone et al. are not suitable for this kind of gameplay, but also that they do not conform to what they define as \<The Good Engagement> (or \<TGE>) rules, i.e. a set of rules that a FPS should satisfy to support and encourage interesting player choices, from which an engaging gameplay should emerge naturally. By analyzing the \<Search \& Destroy>\footnote{A multiplayer game mode where players, divided in two teams, have to eliminate the enemy team or detonate a bomb in their base.} mode of games like \<Counter Strike>\footnote{Valve Software, 2000} and \<Call of Duty>\footnote{Infinity Ward, 2003}, they defined a process to generate suitable maps: starting from a grid, some nodes are selected and connected among them, the result is then optimized to satisfy the TGE rules and finally rooms, resources, objects and spawn points are added, as can be seen in figure \ref{fig:olstedGenerativeProcess}. Opting for an \<interactive> approach, the fitness function used to guide the evolution of these maps is computed on the binary appreciation feedback expressed by real users, since the authors consider bot behavior too different form the one of real users.

\begin{figure}
  \includegraphics[width=\linewidth]{olstedGenerativeProcess}
  \caption{Visual representation of Ølsted et al.\cite{DesignerJob} generative process.}
  \label{fig:olstedGenerativeProcess}
\end{figure}

A completely different approach from the ones listed above is the one of Anand and Wong\cite{10.1007/978-3-662-45212-7_19}, who employed search-based procedural generation to create \<online>, automatically and rapidly multiplayer maps for the \<Capture and Hold>\footnote{A multiplayer game mode where players, divided in two teams, fight for the control of some strategic areas. The score of each team increases over time proportionally to the number of controlled points until one of the two teams reaches a given limit, winning the game.} game mode, without compromising the quality of the final output. To achieve this, they employ a genetic approach, which fitness function is evaluated directly on the topology of the map, considering four different factors: the connectivity between regions, the number of points of collision, the balancing in the positioning of control points and spawn points. With no need to simulate matches, this process can be completed in a matter of seconds. The algorithm starts by generating three maps, that are then evolved by mutation. To obtain the initial maps, Anand and Wong populate a grid with random tiles, they clean it of undesired artifacts and they identify regions within it, that are then populated with strategic points, resources, spawn points and covers. Despite its good results, this approach heavily relies on the validity of the selected topological metrics and, as we have seen, it is still not clear which the good elements of a level are.

\par

Finally, Cachia et al.\cite{MultiLevelEvolution} extended search-based procedural generation to multi-level maps, generating the ground floor with one of the methods defined by Cardamone and employing a random digger for the first floor. The final result can be seen in figure \ref{fig:multiLevelEvolution1}. Their algorithm also positions spawn points and resources through a topological fitness function, which entails the same problems described for Anand and Wong's approach.

\begin{figure}
  \includegraphics[width=\linewidth]{multiLevelEvolution}
  \caption{One of the maps evolved by Cachia's et al.\cite{MultiLevelEvolution} algorithm.}
  \label{fig:multiLevelEvolution1}
\end{figure}

% FPS DESIGN %

\section{History of Level Design in FPS}

\<First Person Shooter> are a video game genre which main features are the first-person player perspective and a weapon-based combat gameplay. During the years this genre has been consonantly evolving, as new elements started to emerge from the very aggressive and fast-paced gameplay of the first games of this kind, like \<Doom>\footnote{ID Software, 1993.}. Games like \<Half-Life>\footnote{Valve Software, 1998.} highlighted the importance of the story and of the setting, games like \<Deus Ex>\footnote{Ion Storm, 2000.} introduced role-play and stealth mechanics and games like \<Quake III: Arena>\footnote{id Software, 1999.} and \<Unreal Tournament>\footnote{Epic Games, 1999.} moved the focus from single player to multiplayer. Finally, Modern Military Shooters introduced a slower and more realistic gameplay and radically changed the setting from fictional conflicts to contemporary ones.

\subsection{Level Design evolution in FPS}

The evolution of the mechanics of this genre was supported by a constant refinement of level design and of the used tools.

\paragraph{Before 1992: The first FPS}

\mbox{}\\

{\setlength{\parindent}{0cm}
The origin of the FPS genre goes back to the beginning of video games themselves. In 1973, Steve Colley developed \<Maze War>, a simple black-and-white multiplayer game set in a tile-based maze, where players would search for other players' avatars, killing them to earn points. Around the same time, Jim Bowery created \<Spasim>, a simple first-person space flight simulator. Both games were never released, instead, we must wait the beginning of the 80's to see the first products available to the public. Heavily inspired by Colley and Bowerey's work, these games had almost no level design, but during the years they started to become a little more complex, with tangled sci-fi structures taking over mazes. 
}

\paragraph{1992: Wolfenstein 3D defines a new genre}

\mbox{}\\

{\setlength{\parindent}{0cm}
When \<Wolfenstein 3D>\footnote{id Software, 1992.} was released in 1992, it changed the genre forever, thanks to its fast gameplay and its light game engine, that allowed to target an audience as wide as possible. Wolfenstein 3D took up the exploitative approach of the period, with its levels full of items, weapons and secret rooms. The level design was still simple, because the technical limitations of the engine and of its \<tile-based>\footnote{The map consists in a grid composed by squared cells, or \<tiles>, of equal size.} \<top-down>\footnote{In \<top-down> games and editors, the game world is seen from above.} level editor allowed to create only flat levels, with no real floor or ceiling and walls always placed at a right angle. 
}

\paragraph{1993 - 1995: Doom and its legacy}

\mbox{}\\

{\setlength{\parindent}{0cm}
A year later, id Software released \<Doom>, a milestone in the history of the genre. The game engine of Doom was capable of many innovations: sections with floor and ceiling with variable height, elevators, non-orthogonal walls, interactive elements, lightning, even dynamic, textures for horizontal surfaces and even a simple \<skybox>\footnote{A method of creating backgrounds that represent scenery in the distance, making the game world look bigger than it really is.}, all of this without losing the speed that characterized Wolfenstein. The developers took advantage as much as possible of the capabilities of the engine, achieving a level design way more complex than what had been seen before. In addition, Doom was designed to be easily modifiable by the users and featured cooperative and competitive multiplayer, via LAN or dial-in connections, that rapidly gathered a massive user base.
}

\par

The impact of Doom on the genre was so strong that in the following years the market was flooded with its clones. All these games, as well as Doom, had still some technical limits, in their engines, that were not completely 3D, in their level editors, that were still top-down and did not allow for the design of more complex levels, and in their gameplay, that still faced limited movements.

\paragraph{1996 - 2000: A constant evolution}

\mbox{}\\

{\setlength{\parindent}{0cm}
In the next years, many games continued what Doom started, bringing constant improvements to the genre. \<Duke Nukem 3D>\footnote{3D Realms, 1996.} set aside the sci-fi settings of its predecessors, switching to real locations, inspired by the ones of Los Angeles. This was possible thanks to its \<2.5D>\footnote{A \<2.5D> engine renders a world with a two dimensional geometry in a way that looks three-dimensional. An additional height component can be introduced, allowing to render different ceiling and floor height. This rendering technique limits the movements of the camera only to the horizontal plane. Doom and many similar games employed this technology.} engine \<Build>\footnote{Developed by Ken Silverman in 1995.}, that provided a \<What You See Is What You Get>\footnote{In computer science, \<What You See Is What You Get > denotes a particular kind of editors where there is no difference form what you see during editing and the final output.} level editor. Furthermore, Build allowed to apply \<scripts> to certain elements of the map, resulting in a more interactive environment.
}

\par

Released in 1996, \<Quake>\footnote{id Software, 1996.} was one of the first and most successful FPS with a real 3D engine, that allowed an incredible jump forward in terms of realism, level design and interactivity. Quake had also a rich multiplayer, with a lot of maps, game modes and special features, like clans and modding. Two years later, \<Unreal>\footnote{Epic Games, 1998.} brought a new improvement in terms of realism and level design, thanks to its engine capable of displaying huge outdoors settings.

\par

In those years many new sub-genres started to spawn, obtained by emphasizing certain features of the previous games or by borrowing mechanics from other genres. \<Quake III Arena>\footnote{id Software, 1999.} and \<Unreal Tournament>\footnote{Epic Games, 1999.} were some of the first and most successful multiplayer games ever released, that required a new approach to level design, completely focused on the creation of competitive maps. Games like \<Deus Ex> introduced tactical and RPG elements, with the possibility of reaching an objective in multiple ways, thanks to a level design that exalted the freedom left to player. Finally, \<Half-Life> changed forever the approach of this genre to storytelling and to level design: the game puts great emphasis on the story, that is narrated from the eyes of the player, without cut-scenes, and introduces the possibility of moving freely between areas, with no interruption. The game also set new standards with its challenging enemy AI, capable of taking advantage of the terrain and coordinating flanking maneuvers. 

\paragraph{2001: The rise of console shooters}

\mbox{}\\

{\setlength{\parindent}{0cm}
In 2001, \<Halo: Combat Evolved>\footnote{Bungie, 2001.} revolutionized the genre, introducing some of the mechanics on which modern FPS are based, as the limited amount of weapons that the player can carry and the regeneration of health over time. This slower and more strategic approach to gameplay matched perfectly with consoles and their twin-stick-controllers, that were not suitable for the extremely fast paced action of the past. Over time, this new approach to gameplay overshadowed almost completely all the others, thanks to the diffusion of consoles and to the increase of production costs, that made PC exclusives economically disadvantageous. From the standpoint of level design, this change required to increase the complexity of the levels and the addition of strategically placed covers.
}

\paragraph{Today: A time of stagnation}

\mbox{}\\

{\setlength{\parindent}{0cm}
Starting from its origins, level design undergone a radical evolution, but in the last years it has shown no significant improvements. This could be due to the considerable risk associated with modern projects or to the lack of suitable instruments.
}

% GRAPH THEORY %

\section{Graph Theory in video games}

\<Graph Theory> has always been used in video games, usually as a useful tool to perform \<pathfinding> for artificial agents. These techniques revolve around the creation of a \<navigation mesh>, i.e. a representation of the walkable areas of a level using non-overlapping polygons obtained by removing the shapes of obstacles from the considered surface. The result is then used to generate a graph, selecting as nodes the vertices or the centers or the centers of edges of the obtained polygons, depending on the kind of movement that must be achieved. This graph can be pre-generated or computed at run-time, if the technique is applied to dynamic environments. Finally, an algorithm like \<A*> is employed to find the shortest path between two points. This process can be seen in figure \ref{fig:pathfinding}

\par

Graphs are also used in procedural generation, as an effective tool to model landmasses and roads.

\begin{figure}
  \includegraphics[width=\linewidth]{pathfinding}
  \caption{A common pathfinding process.}
  \label{fig:pathfinding}
\end{figure}

% SUMMARY %

\section{Summary}

In this chapter we analyzed the current state of level design for First Person Shooters and how this field has been explored in academic research. We then introduced Procedural Content Generation and we observed how some studies have proved that it is a suitable method to produce maps for FPS games. We also took a brief look at the history of First Person Shooters, analyzing how level design evolved over time. Finally, we depicted some of the most common uses of Graph Theory in video games. 

% CHAPTER 3 - UNITY FRAMEWORK %

\chapter{Map design and generation framework}

% INTRODUCTION %

In this chapter we describe the \<framework> that we have developed to study the design and generation of maps for multiplayer Firsts Person Shooters. After a quick overview, we present the map formats that the framework supports and we extensively analyze its structure, its components and its features.

% DESCRIPTION %

\section{Description of the framework}

We designed our framework with the objective of providing a valid alternative to the games currently employed in this research field. All the available options, like \<Cube 2: Sauerbraten>, are powerful tools to perform studies involving artificial agents, but they are not suitable for user-based studies. A data-collection campaign based on these games requires either downloading the game or taking part in real-life play-test sessions, but these options discourage potential participants because they are significantly time-consuming. For this reason, we decided to develop a \<Unity>\footnote{Unity Technologies, 2005. \<Unity> is a game development environment that includes a game editor and a game engine. Currently, it is the most used game development tool.} framework that is as light as possible, with a WebGL build weighting less than 10MB that can be played using any browser.

\par

Since the purpose of this tool is to be used in research, we decided to support most map representation formats used in previous works and we designed our framework to be as modular, expansible and configurable as possible.

% SUPPORTED REPRESENTATION %

\section{Map representation}

Maps are structured as grids of orthogonal \<tiles> and are internally represented  by matrices of characters, where each cell corresponds to a specific \<tile>. Depending on the character it contains, a cell can represent a wall, a floor or an object on the floor. If a cell corresponds to a wall tile we say it is \<filled>, if it corresponds to a floor tile we say it is \<empty>. The framework supports multi-level maps, which are represented by lists of matrices, with each matrix corresponding to a level.

\par

The framework allows to represent maps in two more formats, that are converted to the internal one when provided as input.

% FILE %

\subsection{Text representation}

The text representation allows to encode the map as a text, of which each line corresponds to a row of the internal matrix representation and each character corresponds to a cell. For multi-level maps, the format is the same, with the exception of blank lines used to separate the floors, that are encoded from the lower one to the higher one.

% ALL-BLACK %

\subsection{All-Black representation}

Our All-Black representation is an extended version of the one defined by Cardamone et al.\cite{Cardamone:2011:EIM:2008402.2008411}, to which we have added the support for objects and multi-level maps. In their work, the All-Black representation encodes the empty areas of an otherwise filled map, consisting in square rooms and corridors of fixed width. Rooms are defined by $\langle x,y,s \rangle$ triplets, where $x$ and $y$ define the coordinates of the center of the room and $s$ defines its width. Corridors are rectangular areas with a fixed width of $3$ cells and are defined by $\langle x,y,l \rangle$ triplets, where $x$ and $y$ define the point in which the corridor starts and $l$ defines its length. $l$ also provides the direction of the corridor: if $l$ is positive the corridor extends along the x-axis, otherwise it extends along the y-axis. With respect to this representation, we changed the encoding of the rooms by considering $x$ and $y$ as the coordinates of the corner closer to the origin; this change allows to remove any ambiguity deriving from the position of the center of a room of even width.

\par

For allowing the encoding of objects, we added a third kind of triplet, $\langle x,y,o \rangle$, that uses $x$ and $y$ to denote the coordinates of the tile that hosts the objects and $o$ to denote the object itself, encoded as a character. In our representation, first we store the triplets representing the rooms, then the triplets representing the corridors and finally the triplets representing the objects. These groups are separated by the special character ``$\mid$'' and can have any number of elements, with the exception of the one denoting the rooms, that must have at least one triplet.

\par

We extended the All-Black representation by also including the one defined by Cachia et al.\cite{MultiLevelEvolution}, that allows to encode maps generated with a random digger algorithm, i.e. an algorithm that randomly moves in a filled map emptying all the cells it crosses (for more details see subsubsection \ref{sssec:digger}). The map is encoded by a quintuple $\langle f,l,r,v,s \rangle$, where $f$ encodes the probability of moving forwards, $l$ encodes the probability of turning left, $r$ encodes the probability of turning right, $v$ encodes the probability of jumping to a visited cell and $s$ encodes the probability of placing a flight of stairs, if in a multi-level setting. With respect to the representation defined by Cachia et al., we added the possibility of encoding objects, which group of triplets is separated by the digger quintuple by the special character ``$\mid$''.

\par

Multi-level maps are represented by encoding the floors from the lower one to the higher one using one of the two single-level All-Black formats that we have defined. The encoding of different floors are separated using the double special characters ``$\mid\mid$''.

\par

Figure \ref{fig:allblack} shows two maps with their All-Black representation.

\begin{figure}[tp]
	\centering
	\hfill
  	\begin{subfigure}[t]{0.45\linewidth}
		\includegraphics[width=\linewidth]{ab_divisive_1}
     		\caption{Map represented by $\langle 5, $\ $ 5, $\ $ 9 \rangle $\ $ \langle 10, $\ $ 10, $\ $ 7 \rangle $\ $  \langle 15, $\ $ 25, $\ $ 3 \rangle\ $\ $ \mid $\ $  \langle 5, $\ $ 15, $\ $ 15 \rangle $\ $  \langle 11, $\ $ 15, $\ $ -7 \rangle $\ $  \mid $\ $  \langle 5, $\ $ 5, $\ $ s \rangle $\ $  \langle 7, $\ $ 7, $\ $ d \rangle$.}
 	\end{subfigure}
 	\hfill
  	\begin{subfigure}[t]{0.45\linewidth}
    		\includegraphics[width=\linewidth]{ab_divisive_2}
     		\caption{Map represented by $\langle 1,$\ $2, $\ $5  \rangle $\ $\langle4,$\ $6,$\ $8\rangle $\ $ \mid $\ $ \langle5,$\ $6,$\ $-10\rangle $\ $ \langle10,$\ $15,$\ $-6\rangle $\ $ \mid $\ $ \langle3, $\ $7, $\ $s\rangle $\ $ \langle1, $\ $1, $\ $d\rangle$.}
  	\end{subfigure}
  	\hfill
\caption{Two simple maps with their All-Black representation.}
\label{fig:allblack}
\end{figure}

% FRAMEWORK STRUCTURE % 

\section{Framework structure}

The framework collects data by assigning to the users \<matches> to play. A match is defined by the \<game mode> and by the \<map type>, which in turn is defined by the \<map topology> and by the \<map appearance>. The \<map topology> defines how the map is going to \<be> and depends on the algorithm used to generate it, whereas the \<map appearance> defines how the map is going to \<look> and depends on how the map is assembled. This implies that the map type defines a whole array of procedurally generated maps that share the same topology and appearance. Therefore, when referring to a match we are considering a specific game-mode played in a procedurally generated map. If needed, it is possible to use a pre-generated map instead of generating a new one, by providing it as input using one of the supported formats. In this case the \<map topology> defines how to interpret the input, that is then displayed considering the \<map appearance>.

\par

A match is defined by combining different modules, called \<Managers>, each of which controls a different aspect of the match.

% GAME MANAGER %

\subsection{The Game Manager}

The \<Game Manager> is the module responsible for the overall behavior of a match. Each game mode consists in a different version of the \<Game Manager>. It leans on the \<Map Manager> for the generation and the assembly of the map and on the \<Spawn Point Manager> for the spawn of entities. The \<Game Manager> controls the life-cycle of the match, that can be divided in the following phases:

\begin{itemize}
\item \<Setup>: all the modules are initialized.
\item \<Generation>: the \<Map Manager> generates or imports the map and assembles it.
\item \<Ready>: the \<Game Manager> displays a countdown announcing the start of the game.
\item \<Play>: the \<Game Manager> handles the game while the \<Experiment Manager> logs the actions of the player, if needed. This phase continues until an end condition is satisfied.
\item \<Score>: the \<Game Manager> stops the game and displays the final score.
\end{itemize}

% MAP MANAGER %

\subsection{The Map Manager}

The \<Map Manager> controls the generation, the import and the assembly of the map and the displacement of objects inside it. It leans on the \<Map Generator> for the generation, on the \<Map Assembler> for the \<assembly>\footnote{With \<assembly> we mean the operation of creating a 3D model of the map starting from its matrix representation.} and on the \<Object Displacer> for the \<displacement>\footnote{With \<displacement> we mean the operation of placing the 3D models of the objects in the assembled map, according to their position defined by the \<Map Generator> trough a \<positioning> algorithm.}, whereas it performs the import itself. If the map is provided in input as a text file, the \<Map Generator> is not called, whereas it is used to perform decoding if the map is provided in All-Black format.

\par

The framework provides three different versions of the \<Map Manager>.

\subsubsection{Single-Level Map Manager}

The \<Single-Level Map Manager> is used for any kind of single level map. It can generate maps, import them from file or decode them from All-Black format.

\subsubsection{Multi-Level Map Manager}

The \<Multi-Level Map Manager> is used for any kind of map that has more than one floor. It can generate multi-level maps or import them from file, but it cannot perform All-Black decoding. In addition to the standard modules, it employs a \<Stairs Generator> to position flight of stairs to connect the different floors. 

\par

Multi-level maps are obtained by using at least one generator to produce the desired number of floors. Since this allows to combine different kind of generators, we were able to obtain maps similar to the ones evolved by Cachia et al.\cite{MultiLevelEvolution} (see figure \ref{fig:multilevel_digger}), as well as maps with a more complex and interesting layout than the ones obtained by previous works (see figure \ref{fig:multilevel_divisive}).

\begin{figure}[tp]
	\centering
	\hfill
  	\begin{subfigure}[t]{0.45\linewidth}
	\includegraphics[width=\linewidth]{multilevel_digger}
	\caption{Multilevel map which floors have different topologies.}
	\label{fig:multilevel_digger}
 	\end{subfigure}
 	\hfill
  	\begin{subfigure}[t]{0.45\linewidth}
    			\includegraphics[width=\linewidth]{multilevel_divisive}
	\caption{Multilevel map which floors share the same topology.}
	\label{fig:multilevel_divisive}
  	\end{subfigure}
  	\hfill
\caption{Two multilevel maps.}
\end{figure}

\subsubsection{All-Black Multi-Level Map Manager}

The \<All-Black Multi-Level Map Manager> is used to decode multi-level maps saved in All-Black format. If no stairs are found among the objects, it employs the \<Stairs Generator> to position them.

% MAP GENERATOR %

\subsection{The Map Generator}

The \<Map Generator> controls the generation of the map. Each version of the \<Map Generator> defines a different \<map topology> depending on the used generation algorithm and on how its parametric settings are tuned. Some of these settings are shared by all the versions, whereas some of them are version-specific.

\par

The shared settings are used to define the size of the map and its encoding, to define the objects and to impose some constraints on their positioning:

\begin{itemize}
\item \<Width>: the number of rows of the matrix that represents the map.
\item \<Height>: the number of columns of the matrix that represents the map.
\item \<ObjectToObjectDistance>: the minimum number of cells that must separate two objects. 
\item \<ObjectToWallDistance>: the minimum number of cells that must separate an object and a wall.
\item \<BorderSize>: the width of the border placed all around the map once it has been generated, expressed in number of cells.
\item \<RoomChar>: the character used to represent a clear cell where the player can walk.
\item \<WallChar>: the character used to represent a filled cell where the player cannot walk.
\item \<MapObjects>: a list of the objects that must be placed in the map.
\end{itemize}

\noindent The objects contained in \<MapObjects> can represent spawn points, resources or decoration. They have the following properties:

\begin{itemize}
\item \<ObjectChar>: the character used to represent the object.
\item \<NumObjPerMap>: the number of objects of that kind that must be placed in the map.
\item \<PlaceAnywhere>: if this value is set to true, the restriction on the distance from the walls is ignored.
\item \<PositioningMode>: the algorithm used to position the object in the map.
\end{itemize}

\noindent The framework provides three different algorithms to position the objects inside the map:

\begin{itemize}
\item \<Rain>: positions the objects selecting random cells from the ones that are empty and satisfy the \<ObjectToWallDistance> constraint.
\item \<Rain Shared>: positions the objects selecting random cells from the ones that are empty and satisfy the \<ObjectToWallDistance> constraint and the \<ObjectToObjectDistance> constraint on the objects that have been placed using \<Rain Shared>.
\item \<Rain Distanced>: positions the objects selecting random cells from the ones that are empty and satisfy the \<ObjectToWallDistance> constraint and the \<ObjectToObjectDistance> constraint on the objects with the same \<ObjectChar>.
\end{itemize}

All of the following versions of the \<Map Generator> are deterministic, since they require a \<seed value> as input that constrains the output to a specific map.

% CELLULAR GENERATOR %

\subsubsection{Cellular Generator}

The \<Cellular Generator> employs a parametric \<cellular automaton>\footnote{A \<cellular automaton> consists of a grid of cells, each in one of a finite number of states, such as on and off. For each cell, a set of cells called its neighborhood is defined, usually composed by the ones that share at least one vertex with it (referred as \<8-neighbors>). Given the current state of the grid, a new generation is created, according to some fixed rule that determines the new state of each cell depending on the current state of the cell and of its neighbors.} to generate a natural looking map. 

\par

The algorithm starts by filling some tiles of the map selected at random, then it applies the cellular automaton for a certain number of generations and finally it performs some refinements (for more details, see algorithm \ref{alg:cellular}). The resulting topology depends on the following parameters:

\begin{itemize}
\item \<RandomFillPercent>: the percentage of tiles that are randomly filled during the initialization of the algorithm. High values promote narrow spaces, small values promote wide areas.
\item \<SmoothingInteration>: the number of generations the cellular automaton is ran for. High values penalize small features and make the walls smoother. 
\item \<NeighbourTileLimitLow>: the maximum number of neighbors a cell must have to became empty. Its value must be lesser or equal than the one of \<NeighbourTileLimitHigh>. The map becomes noisier the more they diverge.
\item \<NeighbourTileLimitHigh>: the minimum number of neighbors a cell must have to became filled.
\item \<WallThresholdSize>: the minimum number of cells that an isolated filled region must include to not be deleted. High values penalize small filled regions.
\item \<RoomThresholdSize>: the minimum number of cells that an isolated void region must include to not be deleted. High values penalize small empty regions.
\item \<PassageWidth>: the width of a passage connecting two different areas, expressed in number of cells.
\end{itemize}

\noindent Figure \ref{fig:cellulars} shows how these parameters influence the topology of a map.

\par

The \<Cellular Generator> can perform import and export using the text representation.

% DIVISIVE GENERATOR %

\subsubsection{Divisive Generator}\label{sssec:digger}

The \<Divisive Generator> employs a \<binary space partitioning algorithm> to generate a man-made looking map.

\par

The algorithm starts by obtaining partitions of the map by recursively dividing it in two sides of random size along one of the axes, then it selects some of these partitions as rooms and finally it connects them with corridors (for more details, see algorithm \ref{alg:divisive}). The resulting topology depends on the following parameters:

\begin{itemize}
\item \<RoomDivideProbability>: probability of a partition being divided again. High values promote small rooms.
\item \<MapRoomPercentage>: minimum percentage of tiles of the map that must be empty. High values promote close rooms separated by walls, low values promote distant rooms connected by corridors.
\item \<DivideLowerBound>: minimum division point expressed as percentage of the dimension of the room.
\item \<DivideUpperBound>: maximum division point expressed as percentage of the dimension of the room.
\item \<MinimumRoomDimension>: minimum width expressed in number of cells that a partition must have to be divided again. High values promote large rooms.
\item \<MinimumDepth>: the minimum number of recursive divisions that each partition must have experienced.
\item \<PassageWidth>: the width expressed in number of cells of the corridors connecting the rooms.
\item \<MaxRandomPassages>: the number of additional corridors to place, if possible, once that all the rooms are connected.
\end{itemize}

\noindent Figure \ref{fig:divisives} shows how these parameters influence the topology of a map.

\par

The \<Divisive Generator> can perform both import and export using the text representation, whereas the All-Black format is used only for export. The latter matches perfectly with this generator, since both are based on the concept of rooms and corridors.

% DIGGER GENERATOR &

\subsubsection{Digger Generator}

The \<Digger Generator> employs a simple algorithm to generate a man-made looking map.

\par

The algorithm is iterative and its state is defined by the current cell and by the current direction, that together with a randomly selected action determine the next cell that the algorithm is going to visit. Starting from the central cell of a filled map, at each iteration the algorithm empties the current cell and randomly decides if moving forward, turning left, turning right, jumping to a random visited cell or placing a flight of stairs, if controlled by a \<Multi-Level Generator>. The algorithm stops when a certain percentage of cells has been emptied. The resulting topology depends on the following parameters:

\begin{itemize}
\item \<ForwardProbability>: probability of moving forward in the next iteration. High values promote long corridors. 
\item \<LeftProbability>: probability of moving leftward in the next iteration. High values promote wide areas. 
\item \<RightProbability>: probability of moving rightward in the next iteration. High values promote wide areas. 
\item \<VisitedProbability>: probability of jumping to a visited cell in the next iteration. High values promote a more complex topology. 
\item \<StairProbability>: probability of placing a flight of stairs.
\item \<RoomPercentage>: percentage of tiles of the map that must be empty.
\end{itemize}

\noindent Figure \ref{fig:diggers} shows how these parameters influence the topology of a map.

\par

The \<Digger Generator> can perform both import and export using the text representation, whereas its own All-Black format is used only for import.

% GENERATORS ALGORITHMS AND IMAGES  %

\begin{algorithm}[tp]
\SetAlgoLined
\caption{Cellular generation algorithm.}
\algorithmfootnote{This algorithm is a modified version of the one proposed by Sebastian Lague\cite{lague}.}
\label{alg:cellular}

\For{every cell in the map}{
	empty the current cell\;
}

\While{percentage of filled cells $<$ RandomFillPercent} {
	select a random cell\;
	fill the selected cell\;
}

\For{generation from 0 to SmoothingIterations} {
	\For{every cell in the map}{
		count the 8-neighbors of the cell\;
		\If{8-neighbors  count $>$ NeighbourTileLimitLow}{
  			mark the current cell as filled for the next generation\;
   		}
		\If{8-neighbors  count $<$ NeighbourTileLimitHigh }{
  			mark the current cell as empty for the next generation\;
   		} 
	}
	update the map to the next generation\;
}

\For{every isolated region of empty cells}{
	\If{\#cells in the region $<$ RoomThresholdSize}{
		fill all the cells in the region\;
	}
}

\For{every isolated region of filled cells}{
	\If{\#cells in the region $<$ WallThresholdSize}{
		empty all the cells  in the region\;
	}
}

connect all the regions composed by empty cells\;
place the objects\;

\end{algorithm}

\begin{figure}[tp]
	\centering
  	\begin{subfigure}[t]{0.315\linewidth}
		\includegraphics[width=\linewidth]{cellular_default}
     		\caption{Cellular map generated with the default settings.}
 	\end{subfigure}
 	\hfill
  	\begin{subfigure}[t]{0.315\linewidth}
    		\includegraphics[width=\linewidth]{cellular_rfp40}
    		\caption{Cellular map generated with $Ran\-dom\-Fill\-Per\-cent = 40\%$.}
  	\end{subfigure}
  	\hfill
	\begin{subfigure}[t]{0.315\linewidth}
    		\includegraphics[width=\linewidth]{cellular_rfp50}
    		\caption{Cellular map generated with $Ran\-dom\-Fill\-Per\-cent = 50\%$.}
  	\end{subfigure}
  	
  	\begin{subfigure}[t]{0.315\linewidth}
		\includegraphics[width=\linewidth]{cellular_si0}
     		\caption{Cellular map generated with $Smooth\-ing\-It\-er\-a\-tions = 0$.}
 	\end{subfigure}
  	\hfill
  	\begin{subfigure}[t]{0.315\linewidth}
    		\includegraphics[width=\linewidth]{cellular_si3}
     		\caption{Cellular map generated with $Smooth\-ing\-It\-er\-a\-tions = 3$.}
  	\end{subfigure}
  	\hfill
  	\begin{subfigure}[t]{0.315\linewidth}
    		\includegraphics[width=\linewidth]{cellular_wts5}
     		\caption{Cellular map generated with $Wall\-Thresh\-old\-Size = 5$.}
  	\end{subfigure}	
	\caption[Six maps generated by the Cellular Generator using ``\<ANotSoRandomSeed>'' as seed, but different settings.]{Six maps generated by the Cellular Generator using ``\<ANotSoRandomSeed>'' as seed, but different settings. By default, the Cellular Generator has \<RandomFillPercent> set to $45\%$, \<SmoothingIterations> set to $2$,  \<NeighbourTileLimitHigh> set to $4$,  \<NeighbourTileLimitLow> set to $4$,  \<WallThresholdSize> set to $40$ and \<RoomThresholdSize> set to $100$.}
	\label{fig:cellulars}
\end{figure}

\begin{algorithm}[tp]
\SetAlgoLined
\caption{Divisive generation algorithm.}
\label{alg:divisive}

\For{every cell in the map}{
	fill the current cell\;
}

initialize the partitions list\;
DivideRoom(map, 0)\;

\While{percentage of empty tiles $<$ MapRoomPercentage}{
	extract a partition from the partitions list at random\;
	make the partition a room\;
	empty the tiles in the room\;
}

connect the rooms; 

\While{all the rooms are not directly connected \And \#placed additional corridors $<$ MaxRandomPassages}{
	add an additional corridor between two rooms selected at random;
}

place the objects\;

\hrulefill

\SetKwProg{Fn}{Function}{ is}{end}
\Fn{DivideRoom(section, depth)}{
	\eIf{(true with probability roomDivideProbability \And partition width $>$  minimumDividableRoomDimension \And partition heigth $>$ minimumDividableRoomDimension) \Or depth $<$  minimumDepth} {
		\eIf{previous division was horizontal} {
 			perform a random vertical division between \<divideLowerBound> and \<divideUpperBound>\;
		} {
 			perform a random horizontal division between \<divideLowerBound> and \<divideUpperBound>\;
		}
		DivdeRoom(first sub-section, depth + 1)\;
		DivdeRoom(second sub-section, depth + 1)\;
	}{
		add the partition to the partitions list;
	}	
}

\end{algorithm}

\begin{figure}[tp]
	\centering
  	\begin{subfigure}[t]{0.315\linewidth}
		\includegraphics[width=\linewidth]{divisive_default}
     		\caption{Divisive map generated with the default settings.}
 	\end{subfigure}
	\hfill
  	\begin{subfigure}[t]{0.315\linewidth}
    		\includegraphics[width=\linewidth]{divisive_rdp20}
    		\caption{Divisive map generated with $Room\-Di\-vide\-Prob\-a\-bil\-i\-ty = 20\%$.}
  	\end{subfigure}
	\hfill
  	\begin{subfigure}[t]{0.315\linewidth}
    		\includegraphics[width=\linewidth]{divisive_md1}
    		\caption{Divisive map generated with $Min\-i\-mum\-Depth = 1$.}
  	\end{subfigure}
  	
  	\begin{subfigure}[t]{0.315\linewidth}
		\includegraphics[width=\linewidth]{divisive_md8}
     		\caption{Divisive map generated with $Min\-i\-mum\-Depth= 8$.}
 	\end{subfigure}
 	\hfill
  	\begin{subfigure}[t]{0.315\linewidth}
    		\includegraphics[width=\linewidth]{divisive_mrd1}
     		\caption{Divisive map generated with $Min\-i\-mum\-Room\-Di\-men\-sion = 1$.}
  	\end{subfigure}
	\hfill
  	\begin{subfigure}[t]{0.315\linewidth}
    		\includegraphics[width=\linewidth]{divisive_mrd7}
     		\caption{Divisive map generated with $Min\-i\-mum\-Room\-Di\-men\-sion = 7$.}
  	\end{subfigure}	
	\caption[Six maps generated by the Divisive Generator using ``\<AModeratelyRandomSeed>'' as seed, but different settings.]{Six maps generated by the Divisive Generator using ``\<AModeratelyRandomSeed>'' as seed, but different settings. By default, the Cellular Generator has \<Room\-Di\-vide\-Prob\-a\-bil\-i\-ty> set to $80\%$, \<Map\-Room\-Per\-cent\-age> set to $90\%$,  \<Di\-vide\-Low\-er\-Bound> set to $10\%$,  \<Di\-vide\-Up\-per\-Bound> set to $90\%$,  \<Min\-i\-mum\-Room\-Di\-men\-sion> set to $3$, \<Min\-i\-mum\-Depth> set to $4$, \<Pas\-sage\-Width> set to $3$ and \<Max\-Ran\-dom\-Pas\-sages> set to $12$.}
	\label{fig:divisives}
\end{figure}

\begin{figure}[tp]
\centering
\begin{subfigure}[t]{0.48\linewidth}
\includegraphics[width=\linewidth]{digger_default}
\caption{Digger map generated with the default settings.}
\end{subfigure}
\hfill
\begin{subfigure}[t]{0.48\linewidth}
\includegraphics[width=\linewidth]{digger_roomp20}
\caption{Digger map generated with $Room\-Per\-cent\-age = 20\%$.}
\end{subfigure}

\begin{subfigure}[t]{0.48\linewidth}
\includegraphics[width=\linewidth]{digger_fp60}
\caption{Digger map generated with $For\-ward\-Prob\-a\-bil\-i\-ty = 60\%$, $Right\-Prob\-a\-bil\-i\-ty = 19\%$ and $Left\-ward\-Prob\-a\-bil\-i\-ty = 19\%$.}
\end{subfigure}
\hfill
\begin{subfigure}[t]{0.48\linewidth}
\includegraphics[width=\linewidth]{digger_fp96}
\caption{Digger map generated with $For\-ward\-Prob\-a\-bil\-i\-ty = 96\%$, $Right\-Prob\-a\-bil\-i\-ty = 1\%$ and $Left\-ward\-Prob\-a\-bil\-i\-ty = 1\%$.}
\end{subfigure}
\caption[Four maps generated by the Digger Generator using ``\<AFairlyRandomSeed>'' as seed, but different settings.]{Four maps generated by the Digger Generator using ``\<AFairlyRandomSeed>'' as seed, but different settings. By default, the Digger Generator has \<For\-ward\-Prob\-a\-bil\-i\-ty> set to $90\%$, \<Left\-Prob\-a\-bil\-i\-ty> set to $4\%$, \<Right\-Prob\-a\-bil\-i\-ty> set to $4\%$, \<Vis\-it\-ed\-Prob\-a\-bil\-i\-ty> set to $2\%$, \<Stair\-Prob\-a\-bil\-i\-ty> set to $0\%$ and \<Room\-Per\-cent\-age> set to $50\%$.}
\label{fig:diggers}
\end{figure}

% ALL-BLACK GENERATOR &

\subsubsection{All-Black Generator}

This simple generator parses inputs expressed in All-Black format, extracting rooms and corridors. If no objects are specified, it adds them to the map.

% STAIRS GENERATOR %

\subsection{The Stairs Generator}

The \<Stairs Generator> places stairs in the map after having analyzed it to find possible positions, but if stairs have already been placed by the \<Map Generator> (this happens with the \<Digger Generator>), it just validates them.

% MAP ASSEMBLER %

\subsection{The Map Assembler}

The \<Map Assembler> controls the assembly of the map. Each version of the Map Assembler corresponds to a different \<map appearance>.

% MESH ASSEMBLER %

\subsubsection{Mesh Assembler}

The \<Mesh Assembler> produces a 3D model of the map using an implementation of the \<marching squares algorithm>\footnote{\<Marching squares > is a computer graphics algorithm that generates contours for a \<two-dimensional scalar field>, i.e. a rectangular array of individual numerical values.} to generate three meshes: one for the floor, one for the walls and one for the ceiling. As it can be seen in figure \ref{fig:cellular_assembled}, the result is a natural-looking environment.

% PREFAB ASSEMBLER %

\subsubsection{Prefab Assembler}

The \<Prefab Assembler> produces a 3D model of the map by associating to each tile a specific 3D model, or \<prefab>, depending on the value of the tile and of its 8-neighbors (see figure \ref{fig:prefabs}). Figure \ref{fig:divisive_assembled} shows a map assembled with this algorithm.

\begin{figure}
\centering
\includegraphics[width=0.95\linewidth]{prefabs}
\caption[Some prefabs and the masks they are associated to.]{Some prefabs and the masks they are associated to. Each model refers to the central cell of the corresponding mask. Green denotes empty cells, red denotes filled cell, half-green and half-red denotes cells that are ignored by the mask. The masks can be rotated to obtain all the possible configurations.}
\label{fig:prefabs}
\end{figure}

% MULTI-LEVEL PREFAB ASSEMBLER %

\subsubsection{Multi-Level Prefab Assembler}

Like the \<Prefab Assembler>, the \<Multi-Level Prefab Assembler> produces a 3D model of the map by combining prefabs, but it employs additional logic to manage the overlap of multiple floors. Figure \ref{fig:multi_assembled} shows a map assembled with this algorithm.

\begin{figure}
\centering
\begin{subfigure}[t]{0.48\linewidth}
\includegraphics[width=\linewidth]{isometric_canyon}
\caption{Map generated with the \<Cellular Generator> and assembled with the \<Mesh Assembler>}
\label{fig:cellular_assembled}
\end{subfigure}
\hfill
\begin{subfigure}[t]{0.48\linewidth}
\includegraphics[width=\linewidth]{isometric_mine}
\caption{Map generated with the \<Digger Generator> and assembled with the \<Mesh Assembler>}
\label{fig:digger_assembled}
\end{subfigure}

\begin{subfigure}[t]{0.48\linewidth}
\includegraphics[width=\linewidth]{isometric_factory}
\caption{Map generated with the \<Divisive Generator> and assembled with the \<Prefab Assembler>}
\label{fig:divisive_assembled}
\end{subfigure}
\hfill
\begin{subfigure}[t]{0.48\linewidth}
\includegraphics[width=\linewidth]{isometric_suburbs}
\caption{Multi-level map assembled with the \<Prefab Assembler>}
\label{fig:multi_assembled}
\end{subfigure}
\caption{Some possible combinations of generators and assemblers.}
\end{figure}

% SPAWN POINT MANAGER %

\subsection{The Spawn Point Manager}

The \<Spawn Point Manager> contains a list of all the spawn points displaced in the map, that is populated at the end of the \<generation phase> by the \<Game Manager>. When the \<Game Manager> needs to spawn an entity, the \<Spawn Point Manager> provides a random spawn point from the ones that have not been used in a certain amount of time. If no spawn point meets this condition, the extraction is performed from the complete pool.

% OBJECT DISPLACER %

\subsection{The Object Displacer}

The \<Object Displacer> associates a character that represents neither a wall or a clear cell to the corresponding object, displacing it at the coordinates defined by its position in the map matrix. During this process, it populates a dictionary containing all the objects in the map divided by category, that is used by the \<Game Manager> to populate the list of spawn points used by the \<Spawn Point Manager>.

% EXPERIMENT MENAGER %

\subsection{The Experiment Manager}

The \<Experiment Manager> is a stand alone module that allows to create and manage the experiments used to perform user-based validation. Once that an experiment has been defined, the \<Experiment Manager> automatically assigns to the users the matches to play and collects the desired information.

\paragraph{Experiment definition}

\mbox{}\\

{\setlength{\parindent}{0cm}
An experiment is defined by a \<tutorial>, a list of \<studies> and a \<survey>.}

\par

The \<tutorial> is optional and consists in a match with a simple objective used to explain the commands to the user.

\par

The \<studies> are not optional and each one of them consists in a list of \<cases>. Each case contains a pool of maps and a single game mode, that is used to play the maps in the pool. The maps in the pool are the object of validation, whereas the game mode is the employed validation method.

\par

The \<survey> is optional and consists in a list of multiple-choice questions that are presented to the player at the end of the experiment.

\par

All these elements can be easily customized, as well as the number of test cases that an user has to play in a single experiment session, defined by the parameter \<CasesPerUsers>. The \<Experiment Manager> also allows to diagonally flip the maps, which is an useful method to avoid the rise of a bias due to memorization when the player is presented with different versions of the same map.

\paragraph{Experiment management}

\mbox{}\\

{\setlength{\parindent}{0cm}
Once that the experiment has been defined, it is ready to be played  by the users.}

\par

Each time that a user participates in the experiment, the \<Experiment Manager> selects the least played case of the least played study, in a round-robin fashion. This allows to have equally distributed data for each study and is possible thanks to the completion tracking provided by the \<Experiment Manager> itself. Then, for each case that the user is going to play, a pre-generated map is extracted from the pool and presented to the player as a match of the game mode specified by the case. In a complete experiment, the player will consecutively play the tutorial, one ore more matches and finally answer the survey.

\par

The experiment can be performed \<offline> or \<online>. In the former, the computed data and the experiment completion are stored locally, whereas in the latter, they are stored on a server. If the experiment is provided via an executable, then it is possible to configure it as \<offline>, \<online> or both, with the completion that is stored on a server and the computed data that is stored both locally and remotely. If the experiment is provided via a web build playable via browser, the only supported configuration is the \<online> one. 

\paragraph{Logging}

\mbox{}\\

{\setlength{\parindent}{0cm}
By default, the \<Experiment Manager> produces a complete log of each match, saving the following information:}

\begin{itemize}
\item \<MapInfo>: this field contains general information about the map featured in the match, as its name, its dimension, the size of its tiles and if it has been flipped.
\item \<GameInfo>: this field contains general information about the match, as the experiment name, the game mode and the duration.
\item \<SpawnLogs>: this field contains a list of all the spawn events. Each entry contains a timestamp, the coordinates of the spawn point and the name of the spawned entity.
\item \<PositionLogs>: this field contains a discretized list of the positions occupied by the player during the match, acquired with a given frequency. Each entry contains a timestamp, the coordinates of the player and the direction he is facing expressed in degrees.
\item \<ShotLogs>: this field contains a list of all the shots fired by the player. Each entry contains the same fields of the \<PositionLogs>, plus the identifier of the firing weapon, the number of projectiles in its magazine and its total available ammunition.
\item \<ReloadLogs>: this field contains a list of all the reloadings performed by the player. Each entry contains a timestamp, the identifier of the weapon that is being reloaded, the number of projectiles in its magazine and its total available ammunition, both before the reloading.
\item \<HitLogs>: this field contains a list of all the shots that hitted an entity. Each entry contains a timestamp, the coordinates of the hitted entity, the name of the hitted entity, the name of the hitter entity and the caused damage.
\item \<KillLogs>: this field contains a list of all the killings. Each entry contains a timestamp, the coordinates of the killed entity, the name of the killed entity and the name of the killer entity.
\end{itemize}

It is possible to customize the \<Experiment Manager> to have it compute and save specific metrics in a different log. Moreover, if the experiment includes a survey, the answers of the user are saved in a dedicated log.

\paragraph{Data retrieval}

\mbox{}\\

{\setlength{\parindent}{0cm}
The framework provides a simple interface for downloading the logs stored on the server. Since it is possible to set a limit on the dimension of logs which causes them to be split in multiple parts, the framework automatically performs merging and signals incomplete logs.
}

% ENTITIES %

\section{Entities}

The \<entities> are the \<agents> that take part in a match. All the entities share the following common features:

\begin{itemize}
\item \<TotalHealth>: the maximum number of health points of the entity, i.e. the quantity of damage the entity can receive before being destroyed.
\item \<Guns>: the guns associated to the entity.
\end{itemize}

The framework includes three different kind of agents:

\begin{itemize}
\item \<Player>: the entity that is controlled by the user. It can walk, jump, aim, deal and receive damage and pick resources.
\item \<Opponent>: this entity is similar to the one of the \<Player>, but in the current version of the framework it has no active logic, beside the one that controls its health.
\item \<Target>: this simple entity rotates in place. Besides receiving damage, it can harm the player thanks to the laser gun it can be equipped it. Figure \ref{fig:targets} shows different kind of targets.
\end{itemize}

\begin{figure}
\centering
\includegraphics[width=0.95\linewidth]{targets}
\caption[Different ready-to-use target entities provided by the framework.]{Different ready-to-use target entities provided by the framework. From the first to the third are simple targets, from the fourth to the sixth are targets equipped with two opposing laser guns, from the seventh to the ninth are``\<core>'' targets equipped with an increasing number of radial laser guns. The size of each target is proportional to its \<TotalHealth>.}
\label{fig:targets}
\end{figure}

% WEAPONS %

\section{Weapons}

The framework allows to easily define any kind of fire arm starting from a common parametric structure that characterizes the basic behavior of a gun with the following variables: 

\begin{itemize}
\item \<Damage>: the damage inflicted by a single projectile.
\item \<Dispersion>: the aperture of the cone-shaped projectile spread expressed in degree.
\item \<ProjectilePerShot>: the number of projectile emitted with one shot.
\item \<InfiniteAmmo>: tells if the gun has infinite ammunition.
\item \<ChargerSize>: the capacity of the gun magazine.
\item \<MaximumAmmo>: the maximum quantity of ammunition that can be carried for a specific gun.
\item \<ReloadTime>: the amount of time needed to reload the gun.
\item \<CooldownTime>:  the amount of time needed after a shot to fire again.
\item \<AimEnabled>: tells if the gun allows the player to aim.
\item \<Zoom>: the zoom provided by the scope when aiming.
\end{itemize}

\noindent This parametric approach allows to use the framework as a tool for user-based validation of procedurally generated weapons, that is another research field that has been explored in recent years \cite{ArmiProcedurali}.

\par 

The framework comes with three different categories of weapons already implemented.

\paragraph{Raycast guns}

\mbox{}\\

{\setlength{\parindent}{0cm}
\<Raycast guns> are weapons which projectiles have no \<time of flight>, but instantly hit the target once shot. The only additional parameter that this category introduces is \<Range>, which can be used to limit the reach of the weapon.
}

\par

There are three weapons of this category that the player can use:

\begin{itemize}
\item \<Assault Rifle>: a medium range weapon with a high fire rate, a capacious magazine and no dispersion which shots single medium damage projectiles. Figure \ref{fig:assault} shows its model and table \ref{tab:gunconfig} shows its parameters.
\item \<Shotgun>: a short range weapon with a slow fire rate, a small magazine and high dispersion which shots multiple low damage projectiles. Figure \ref{fig:shotgun} shows its model and table \ref{tab:gunconfig} shows its parameters.
\item \<Sniper Rifle>: a long range weapon with a slow fire rate, a small magazine and no dispersion which shots single high damage projectiles. It is equipped with a scope. Figure \ref{fig:sniper} shows its model and table \ref{tab:gunconfig} shows its parameters.
\end{itemize}

\paragraph{Projectile guns}

\mbox{}\\

{\setlength{\parindent}{0cm}
\<Projectile guns> are weapons that shoot projectiles with a limited flight speed. This category introduces two additional parameters:
}

\begin{itemize}
\item \<ProjectileLifetime>: if the projectile does not hit anything after this amount of time, it is destroyed.
\item \<ProjectileSpeed>: the speed of the projectile.
\end{itemize}

The only weapon of this category that the player can use is the \<Rocket Launcher>, a long range weapon with a slow fire rate, a small magazine and no dispersion which shots explosive projectiles. The projectiles of this weapon are slow and explode on impact, dealing an high damage that decreases radially from the center of the explosion. Figure \ref{fig:rocket} shows its model and table \ref{tab:gunconfig} shows its parameters.

\begin{figure}[p]
\centering
\begin{subfigure}[t]{0.48\linewidth}
\includegraphics[width=\linewidth]{gun_assault}
\caption{The Assault Rifle.}
\label{fig:assault}
\end{subfigure}
\begin{subfigure}[t]{0.48\linewidth}
\includegraphics[width=\linewidth]{gun_shotgun}
\caption{The Shotgun.}
\label{fig:shotgun}
\end{subfigure}
\begin{subfigure}[t]{0.48\linewidth}
\includegraphics[width=\linewidth]{gun_rocket}
\caption{The Rocket Launcher.}
\label{fig:rocket}
\end{subfigure}
\begin{subfigure}[t]{0.48\linewidth}
\includegraphics[width=\linewidth]{gun_sniper}
\caption{The Sniper Rifle.}
\label{fig:sniper}
\end{subfigure}
\caption{The weapons that the player can use.}
\end{figure}

\begin{table}[p]
\setlength\extrarowheight{2pt}
\begin{tabularx}{\textwidth}{|l|C|C|C|C|}
\cline{2-5}
\multicolumn{1}{c|}{}& Assault Rifle & Shotgun & Rocket Launcher & Sniper Rifle  \\
\hline
\<Damage> & 15  & 20  &  120 & 75  \\
\hline
\<Dispersion> & 0  & 7.5  & 0  & 0  \\
\hline
\<ProjectilesPerShot> &  1 &  5 & 1  & 1  \\
\hline
\<InfiniteAmmo> & false & false  & false &   false\\
\hline
\<ChargerSize> & 32  & 3  &  2 &  5 \\
\hline
\<MaximumAmmo> &  120 &  24 & 16  & 30  \\
\hline
\<ReloadTime> &  1 & 1  &  1 & 1  \\
\hline
\<CooldownTime> &  0.1 &  0.75 &  0.75 &  0.5 \\
\hline
\<AimEnabled> &  false & false  & false  & true  \\
\hline
\<Zoom> & 1  &  1 &  1 & 3  \\
\hline
\<LimitRange> & false  & true  &  - & false  \\
\hline
\<Range> &  - &  100 &  - &  - \\
\hline
\<ProjectileLifeTime> & -  & -  & 10  & -  \\
\hline
\<ProjectileSpeed> &  - &  - & 50 &  - \\
\hline
\end{tabularx}
\caption{Parametric configuration of the four weapons available to the player.}
    \label{tab:gunconfig}
\end{table}

\paragraph{Laser guns}

\mbox{}\\

{\setlength{\parindent}{0cm}
\<Laser guns> are not based on the same structure of the previous categories. Laser guns emit a continuous ray that deals damage over time to everything it touches. Their only configurable parameter is \<DPS> (\<damage per second>), i.e. the damage that the gun deals in a second when continuously hitting a target.
}

% OBJECTS %

\section{Objects}

Beyond \<decorations>, that are simple 3D models with no logic used to graphically enrich the map, the framework provides \<spawners>, i.e. objects that spawn a resource that can be collected by the entities. Once that the resource is collected, it disappears for an interval  of time defined by the parameter \<Cooldown>. The framework comes with two different \<spawners>:

\begin{itemize}
\item \<Health pack Spawner>: it spawns health packs, that partially restore the health of the entity. The healed amount of health is defined by \<RestoredHealth>.
\item \<Ammunition Spawner>: it spawns ammunition crates, that supply the entity with ammunition. \<SuppliedGuns> defines which guns the crates can supply, whereas \<AmmoAmounts> defines how many ammunition are provided to the entity for each supplied weapon.

\end{itemize}

% GAME MODES %

\section{Game modes}

The framework comes with three different game modes, that have been designed to highlight specific aspects of a multiplayer FPS. Each game mode is defined by a different version of the \<Game Manager>.

\subsection{Duel}

The \<Duel> game mode is a classic \<deathmatch> redistricted to two entities, with one of the two being the player. Each time that an entity eliminates the other one, it scores one point, whereas it loses one if it destroys itself by accident. When an entity has been eliminated, it \<respawns>\footnote{In video games, \<respawn> denotes the reappearing in a specific location, called \<spawn point>, of an entity which has been eliminated.} at a random spawn point. At the end of the match, which is marked by a time limit, the winner is the contender who has scored the highest number of points.

\par

Of the game modes that the framework provides, this one is the most complete, because it contains all the dynamics that characterize a multiplayer FPS match, and the most important, since it is the one that is usually used to perform validation in this research field.

\subsection{Target Rush}

\begin{figure}
\centering
\includegraphics[width=0.95\linewidth]{rush20}
\caption{Targets equipped with laser guns in the final wave of a \<Target Rush> match.}
\label{fig:targets}
\end{figure}

In the \<Target Rush> game mode the player faces increasingly difficult waves of enemies, trying to obtain a score as high as possible before the end of the game, that is triggered by the player death or by the countdown hitting zero. The player earns points and additional time when he destroys an enemy or he completes a wave. The number of waves is parametric, as well as the content of each one. By default, this mode has twenty waves and uses targets as enemies, that start as harmless but became more and more numerous and dangerous with each wave (see figure \ref{fig:targets}). 

\par

This game mode has been designed to force the player to explore the map, through the research of enemies, health packs and ammunition, that quickly become indispensable as the match progresses.

\subsection{Target Hunt}

In the \<Target Hunt> game mode the player needs to find and eliminate a series of enemies in a given amount of time. The  enemies that the player is going to face are stored in a parametric list that is read circularly and are spawned one at a time, as soon as the previous one has been eliminated. To each enemy is assigned a score.

\par

This game mode has been designed to force the player to search a specific objective in the map.

\section{Summary}

In this chapter we analyzed the framework that we have developed to perform user-based validation, focusing on its structure, its components and its parametric nature.

% CHAPTER 4 - PYTHON FRAMEWORK %

\chapter{Graph-based map analysis}

% INTRODUCTION %

In this chapter we describe the approach that we have developed to perform analysis and populating of pre-generated maps using \<Graph Theory>. After a quick overview, we introduce the analysis capabilities of this approach and then we present how we employed it to strategically place game elements in pre-generated maps.

% DESCRIPTION %

\section{Overview}

Our approach consists of generating different kind of graphs, starting from the text and the All-Black representation of a map, that are used to perform various analysis and manipulation operations.

\par

The use of All-Black format is convenient, because it provides by default a logical division of the map in different areas and it allows our approach to be applied by other researchers, since as we have seen the All-Black format is widely used in the literature. In this thesis we focused on positioning objects in pre-generated maps, but, for instance, the same approach could be used to address the identification and definition of design patterns from an unfamiliar perspective or for \<direct evaluation> in Search Based PCG.

% ANALYSIS %

\section{Analysis of the map}

The analysis is performed by generating different kind of undirected graphs, each one used to highlight a different feature of the map in question, using a \<Python> tool based on \<NetworkX>\footnote{A solid graph theory library (\url{https://networkx.github.io/}).} that we developed.

\subsection{Outlines graph}

The \<outlines graph> is generated starting from the All-Black representation of a map and is obtained by associating a node to every vertex of every room and corridor and by connecting the non-adjacent ones that belong to the same outline. This graph has a single kind of node (\<vertex node>) that contains the coordinates of the tile it represents, which are used to position the node when the graph is visualized. Figure \ref{img:graph_out} shows an example of this graph. This graph can be used to visualize the rooms which compose the map.

\subsection{Reachability graphs}

Our tool can generate various kinds of \<reachability graphs> that represent various ways in which it is possible to navigate the map. In these graphs a node represents a reachable position, whereas an edge indicates a viable path from a position to another.

\subsubsection{Tiles graph}

The \<tiles graph> is generated starting from the text representation of a map and is obtained by associating a node to each empty tile and by connecting each node to its 8-neighbors. The horizontal and vertical edges have cost $1$, whereas the diagonal ones have cost $\sqrt{2}$. This graph has a single kind of node (\<tile node>) that contains the coordinates of the tile it represents, which are used to position the node when the graph is visualized. Figure \ref{img:graph_tile} shows an example of this graph. This graph can be used to find the minimum distance that separates two cells, along with the shortest path that connects them.

\subsubsection{Rooms graph}

The \<rooms graph> is generated starting from the All-Black representation of a map and is obtained by associating a node to each room and corridor and by connecting nodes which corresponding rooms or corridors overlap, using as weight the Euclidean distance of their central tile. This graph has a single kind of node (\<room node>), used to represent both rooms and corridors that contains the coordinates of the closest and furthest vertex of the room from the origin. When visualized, each node is positioned on the coordinates of the central tile of the room it represents. Figure \ref{img:graph_room} shows an example of this graph. This graph can be used to analyze the topology of a map, in order to find loops, choke points, central areas and other kind of structures.

\subsubsection{Rooms and game elements graph}

The \<rooms and game elements graph> is an extension of the rooms graph, which also includes game elements as nodes, that are connected to the nodes corresponding to the rooms and corridors which contain them. In addition to the room node inherited form the rooms graph, this graph has a node to represent game elements (\<element node>) that contains the coordinates of the game element, which are used to visualize the node, and the character associated to it. Figure \ref{img:graph_room_res} shows an example of this graph.

\subsection{Visibility graph}

The \<visibility graph> is generated starting from the text representation of a map and is obtained by associating a node to each empty tile and by connecting each node to all the tiles that are visible from that node. For two tiles to be respectively visible, it must be possible to connect them with a line without crossing any filled tile. This graph has a single kind of node (\<visibility node>) that contains the coordinates of the tile it represents, which are used to position the node when the graph is visualized, and its \<visibility>, which is computed as the \<degree centrality>, i.e. the number of edges incident to that node. A tile with high visibility allows to control a wide portion of a map, but at the same time an entity standing on it is easy to spot. To make this graph easier to read by the user, the tool associates a color to the nodes, which ranges from blue, for the ones with the minimum visibility, to red, for the ones with the maximum visibility. This can be seen in figure \ref{img:graph_visibility}. This graph can be used to analyze which areas of the map are more exposed and which ones are more repaired.

\begin{figure}[]
	\centering
	\hfill
  	\begin{subfigure}[t]{0.45\linewidth}
		\includegraphics[width=\linewidth]{graph_divisive}
     		\caption{The map.}
     		\label{img:graph_divisive}
 	\end{subfigure}
 	\hfill
  	\begin{subfigure}[t]{0.45\linewidth}
    		\includegraphics[width=\linewidth]{graph_out}
    		\caption{The outlines graph of the map.}
     		\label{img:graph_out}
  	\end{subfigure}
  	\hfill
  	
  	\hfill
  	\begin{subfigure}[t]{0.45\linewidth}
    		\includegraphics[width=\linewidth]{graph_tile}
    		\caption{The tiles graph of the map.}
     		\label{img:graph_tile}
  	\end{subfigure}
  	\hfill
  	\begin{subfigure}[t]{0.45\linewidth}
    		\includegraphics[width=\linewidth]{graph_room}
    		\caption{The rooms graph of the map.}
     		\label{img:graph_room}
 	\end{subfigure}
 	\hfill
 	
 	\hfill
  	\begin{subfigure}[t]{0.45\linewidth}
    		\includegraphics[width=\linewidth]{graph_room_res}
    		\caption{The rooms and game elements graph of the map.}
     		\label{img:graph_room_res}
  	\end{subfigure}
  	\hfill
  	\begin{subfigure}[t]{0.45\linewidth}
    		\includegraphics[width=\linewidth]{graph_visibility}
    		\caption{The visibility graph of the map.}
     		\label{img:graph_visibility}
  	\end{subfigure}	
  	\hfill
	\caption{A map and all the graphs that can be generated from it.}
\end{figure}

\subsection{Interesting metrics}
\label{ss:interesting_metrics}

Considering the graphs, in particular the ones with room nodes, the following metrics defined by Graph Theory provide interesting information about the layout of a map:

\begin{itemize}
\item \<Degree centrality>: defined for a node, it is the number of edges that the node has. If the node represents a room, it measures how many entrance or exits the room has. 
\item \<Closeness centrality>: defined for a node, it measures its centrality in the graph, computed as the sum of the lengths of the shortest paths between the node and all other nodes in the graph. If the node represents a room, it measures how central the room is.
\item \<Betweenness centrality>: defined for a node, it measures its centrality in the graph, computed as the number of shortest paths connecting the nodes that pass through the node. If the node represents a room, it measures how central the room is.
\item \<Connectivity>: defined for a graph, it is the minimum number of elements (nodes or edges) that need to be removed to disconnect the remaining nodes from each other. If the graph represents a map, it measures the existence of isolated areas.
\item \<Eccentricity>: defined for a node, it is the maximum distance from the node to all other nodes in the graph. If the node represents a room, it measured how isolated the room is.
\item \<Diameter>: defined for a graph, it is the maximum eccentricity of its nodes. If the graph represents a map, it measures the size of the map.
\item \<Radius>: defined for a graph, it is the minimum eccentricity of its nodes. If the graph represents a map, it measures how distanced the rooms are from each other.
\item \<Periphery>: defined for a graph, it is the set of nodes with eccentricity equal to the diameter. If the graph represents a map, it defines its peripheral areas.
\item \<Center>: defined for a graph, it is the set of nodes with eccentricity equal to the radius. If the graph represents a map, it defines its central areas.
\item \<Density>: defined for a graph, it ranges from 0 to 1, going from a graph without edges to a complete graph. If the graph represents a map, it measures how complex it is.
\end{itemize}

% POPULATING %

\section{Placement of game elements on the map}

We have defined multiple heuristics to populate a map with game elements using the metrics that can be extracted from a graph. These heuristic are a mathematical transposition of rules and patterns concerning game elements placement that we have extracted from the work of Tim Schäfer\cite{great1vs1}, who has performed an in depth analysis of multiplayer 1vs1 maps for \<Quake 2>\footnote{Id Software, 1997}.

\subsection{Rules for the placement of game elements}

The balance of a deathmatch game radically changes each time that a player is killed. If the game has more than two players, the player who won the fight does not gain any strategic advantage, since he still has the other players to face, whereas the defeated player is put at considerable disadvantage, because on death he loses all the weapons and ammunition that he has collected. In a 1vs1 match, a kill has an even stronger influence, since the surviving player has more weapons and ammunition and gains the complete control of the map, that comes with the chance of scoring another easy kill, as soon as the other player respawns, or of searching for additional equipment. Schäfer refers to the surviving player as \[up-player] and to the defeated one as \[down-player].

\par

To design a multiplayer map that is interesting and fun to play, it is important to consider the up-player vs down-player dynamic both when defining the map layout and when positioning game elements.

\par

The spawn points, i.e. the locations where the down-player reappears, should be positioned in areas that are of low interest for the up-player and that are easy to leave. Obviously, central hubs and dead ends are a bad choice, whereas rooms with 2 or 3 exits are usually the best option. 

\par

For what concerns the resources, they must be placed considering both the up-player vs down-player dynamic and the characteristics of the resource itself. It is important to place the right amount of resources on the map, because too many would eliminate the need for exploration, whereas too few would disadvantage the down player. It is also important not to place too many powerful items in the same area or in boring spots, since the risk to obtain them should always be proportional to the strategical advantage they allow to achieve. It is important to consider that a powerful resource is interesting for both players, so it often acts as a \<point of collision>. The resources usually are of five kinds: health packs, \<armors>, \<power-ups>, ammunition and weapons. 

\par

The health packs are placed in zones that are safe or not too dangerous. They have no use for the down-player, that respawns with full health, but they can be useful for the up-player, if he has been damaged during the fight, whereas they always come in handy during a fight or after that one of the contenders disengages.

\par

Armors, which supply a second health that is consumed before the main one, are usually placed in spots that are aimed both at the down-player and at the up-player: objects that provide a small quantity of armor should be easy to achieve, whereas the ones that provide full armor should be placed in dangerous areas.

\par

Power-ups grant temporary advantages to the player who collects them, like invisibility or increased damage, and are placed in locations difficult to reach and contextual to their effect.

\par 

The position of a weapon and of its ammunition depends on the weapon itself. We can divide the weapons in three categories: \<weak>, \<medium> and \<strong>. Weak weapons are of a certain interest for the down-player, if he has not collected any other weapon yet, and of no interest for the up-player, so they are placed near spawn-points or in gaps where no other weapon is available, together with their ammunition. Medium weapons are of high interest for the down-player, since he needs to get one of them as soon as possible if he wants to face the up-player, so they are placed in areas that are easy to reach and the same goes for their ammunition. Finally, the strong weapons should be placed in areas that are strategically disadvantageous, like dead ends or vertically dominated areas, or difficult to reach. If a weapon is very contextual, i.e. it is useful in very few situations, it is usually placed in an area that allows to take advantage of its features, whereas a weapon that is strong in almost any situation is usually placed in an area where it cannot be used optimally (e.g. a rocket launcher in a small room).

% PLACEMENT PROCESS % 

\subsection{Placement process}
\label{ss:placement}

Starting from these considerations, we defined a process that allows to position any kind of game element in two steps: the selection of a room and the selection of a tile inside the room. This process can be repeated as many time as needed, after having updated the graphs with the newly added game element.

% ROOM %

\subsubsection{Room selection}
\label{sss:room_selection}

The selection of a room can be heuristic-based, uniform or random.

% HEURISTIC ROOM %

\paragraph{Heuristic-based room selection} 

This method selects rooms considering three suitability criteria:

\begin{itemize}
\item \<Degree heuristic>: defined by the function $D(r)$, where $r$ is a room node, it measures how much the degree centrality of the node matches the desired one.
\item \<Game element closeness heuristic>: defined by the function $H_e(r)$, where $r$ is a room node, it measures how much the closeness of the node to the already placed element nodes matches the desired one.
\end{itemize}

Given the rooms and game elements graph of a map ($G_{rr}$) and the subset of room nodes ($R \subseteq G_{rr}$), the room node which is selected is the one which maximizes the following weighted sum of functions:
%
\begin{align}
r^* = \argmax_{r \in R} (w_D  \times D(r) + w_{H_e}  \times H_e(r))
\label{eq:room_heuristic}
\end{align}
%
\noindent
The weights $w_D$ and $w_{H_e}$ allow to define how much each one of the two heuristics influences the selection of a room. Both functions should be defined to output a value in the range $[0,1]$, in order to have the same influence for equal weight.

% UNIFORM ROOM %

\paragraph{Uniform room selection} 

This method selects rooms that are uniformly distributed in the map. The first room is selected at random, then, given the rooms and corridors graph ($G_{rr}$), the subset of room nodes ($R \subseteq G_{rr}$) and the subset of element nodes ($S \subset G_{rr}$), the remaining rooms are selected with the following heuristic:
%
\begin{align}
	r^* = \argmax_{r \in R} ( \min_{s \in S} (\spl(r, s) ))
	\label{eq:tile_heuristic}
\end{align}
%
\noindent
where $\spl(n, m)$ denotes the length of the shortest path that connects the two nodes $n$ and $m$, found using Dijkstra's algorithm.

% RANDOM ROOM %

\paragraph{Random room selection} 

This method simply selects rooms at random.

% TILE %

\subsubsection{Tile selection}

The selection of a tile can be heuristic-based, uniform or random.

% HEURISTIC TILE %

\paragraph{Heuristic-based tile selection} 

This method selects tiles considering three suitability criteria:

\begin{itemize}
\item \<Visibility heuristic>: defined by the function $v(t)$, where $t$ is a tile node, it measures how much the visibility of the corresponding tile matches the desired one.
\item \<Wall closeness heuristic>: defined by the function $H_w(t)$, where $t$ is a tile node, it measures how much the proximity of the corresponding tile to the walls matches the desired one.
\item \<Game element closeness heuristic>: defined by the function $H_e(t)$, where $t$ is a tile node, it measures how much the proximity of the node with already placed element nodes matches the desired one.
\end{itemize}

Given $G_v$ the visibility graph and $T \subset G_v$ the subset of tiles contained by the selected room, the tile which is selected is the one which maximizes the following weighted sum of functions:
%
\begin{align}
t^* = \argmax_{t \in T} (w_v \times v(t) + w_{h_w}  \times h_w(t) + w_{h_e}  \times h_e(t))
\end{align}
%
\noindent
The weights $w_v $, $w_{h_w}$ and $w_{h_e}$ allow to define how much each one of the three heuristics influences the selection of a tile. All three functions should be defined to output a value in the range $[0,1]$, in order to have the same influence for equal weight.

% UNIFORM TILE %

\paragraph{Uniform tile selection} 

This method selects tiles that are uniformly distributed in the room. If no game element has been placed, the central tile is selected, otherwise it is picked out the one that maximizes the distance from the already positioned game elements but is not too close to the walls.

% RANDOM TILE %

\paragraph{Random tile selection} 

This method simply selects tiles from the room at random.

% HEURISTICS %

\subsection{Heuristics for the placement of game elements}

Depending on how a game element needs to be placed inside the map, we have defined different heuristics to perform heuristic-based selection of rooms and tiles.

% SPAWN POINTS %

\subsubsection{Spawn points}
\label{sss:sph}

For spawn points, the heuristic-based approaches for rooms and tiles respectively select rooms that have a small number of connections, are not dead ends and are distant from each other and tiles that offer the best balance between low visibility and distance from the walls. In this way spawn points are placed in passageways that are sheltered and easy to leave. 

% ROOM %

\paragraph{Room selection}

Considering equation \ref{eq:room_heuristic} and given the rooms and game elements graph of the map ($G_{rr}$), the subset of room nodes ($R \subseteq G_{rr}$) and the subset of element nodes ($S \subset G_{rr}$), the most suitable room for containing a spawn point is selected using the following heuristics:

\begin{align}
\label{eq:lowriskdeg}
D(r) = \begin{cases}
    		\hfil 0 & \text{if } \degcent(r) = 1 \\
    		1 - \cfrac{\degcent(r) - \min_{r' \in R}\degcent(r')}{\max_{r' \in R}\degcent(r') - \min_{r' \in R}\degcent(r')} & \text{if } \degcent(r) \neq 1 \
  	\end{cases}
\end{align}
%
\begin{align}
\label{eq:lowriskres}
H_e(r) = \min_{n \in G_{rr}}
  	\begin{cases}
    		\hfil 1 & \text{if } n \notin S \\
    		\cfrac{\spl(r, n)}{\diam(G_{rr})} & \text{if } n \in S \
  	\end{cases}
\end{align}
%
\noindent
where $\degcent(n)$ denotes the connectivity degree of the node $n$ and $\diam(G)$ the diameter of the graph $G$. Equation \ref{eq:lowriskdeg} promotes rooms with few passages but that are not dead ends (see figure \ref{fig:degree}), whereas equation \ref{eq:lowriskres} promotes rooms that are distant from the already placed game elements. Both are normalized in the range $[0,1]$. We empirically set the weighs to $w_D = 1 $ and $ w_{H_e} = 0.5 $.

\begin{figure}
\begin{minipage}[t]{0.4825\linewidth}
\includegraphics[width=\linewidth]{degree}
\caption[How the \<degree heuristic> defined for spawn points varies depending on the room node degree.]{How the \<degree heuristic> defined for spawn points varies depending on the room node degree, with the degree ranging in $[0, 15]$.}
\label{fig:degree}
\end{minipage}
\hfill
\begin{minipage}[t]{0.4825\linewidth}
\includegraphics[width=\linewidth]{visibility_low}
\caption[How the \<visibility heuristic> defined for spawn points varies depending on the tile node degree.]{How the \<visibility heuristic> defined for spawn points varies depending on the tile node degree, with the degree ranging in $[0, 15]$.}
\label{fig:visibility_low}
\end{minipage}
\end{figure}

% TILE %

\paragraph{Tile selection}

Considering equation \ref{eq:tile_heuristic} and given the visibility graph of the map ($G_v$), the subset of tile nodes that belong to the room ($T \subset G_v$), the subset of tile nodes of the room which contain a game element ($H \subset T$) and the length of the diagonal of the room  ($l_d$), the most suitable tile for containing a spawn point is selected using the following heuristics:
%
\begin{align}
\label{eq:lowvis}
v(t) = 1 - \cfrac{\degcent(t) - \min_{t' \in G_v}\degcent(t')}{\max_{t' \in G_v}\degcent(t') - \min_{t' \in G_v}\degcent(t')}
\end{align}
%
\begin{align}
\label{eq:lowwallclos}
h_w(t) = \cfrac{\walld(t, r^*)}{\max_{t' \in T}\walld(t', r^*)}
\end{align}
%
\begin{align}
\label{eq:lowresclos}
h_e(t) = \begin{cases}
    		\hfil 0 & \text{if } | H | = 0 \\
    		\min_{h \in H}\cfrac{\cartd(t, h)}{l_d}& \text{if } | H | > 0 \
  	\end{cases} 
\end{align}
%
\noindent
where $\walld(n, r)$ is the distance of the coordinates associated to the node $n$ from the walls of the room $r$, computed as the sum of the minimum distances from the horizontal and vertical walls, and $\cartd(n, m)$ is the distance of the coordinates associated to the nodes $n$ and $m$. Equation \ref{eq:lowvis} promotes tiles with low visibility (see figure \ref{fig:visibility_low}), equation \ref{eq:lowwallclos} promotes tiles that are distant from the walls, whereas equation \ref{eq:lowresclos} promotes tiles that are distant from the already placed game elements inside the room. All three are normalized in the range $[0,1]$. We empirically set the weights as $ w_v = 1 $, $ w_{h_w} = 0.5 $ and $ w_{h_e}  = 0.5 $.

% HEALTH PACKS %

\subsubsection{Health packs}

For heath packs, the heuristic-based approaches for rooms and tiles respectively select rooms that have a low-medium number of connections and are distant from each other and tiles that offer the best balance between medium visibility and distance from the walls. In this way the heath packs are placed in areas that are easy to reach and are not too exposed.

% ROOM %

\paragraph{Room selection}

Considering equation \ref{eq:room_heuristic}  and given the rooms and corridors graph of the map ($G_{rr}$), the subset of room nodes ($R \subseteq G_{rr}$) and the subset of element nodes ($S \subset G_{rr}$), the most suitable room for containing a health pack is selected using the following heuristic:
%
\begin{align}
\label{eq:lowriskdegh}
D(r) = 1 - \cfrac{f_{deg}(r) - \min_{r' \in R}f_{deg}(r')}{\max_{r' \in R}f_{deg}(r') - \min_{r' \in R}f_{deg}(r')} 
\end{align}
%
\noindent
with
%
\begin{align}
\label{eq:degfit}
f_{deg}(r) = \intd(\degcent(r),0.3,0.5)
\end{align}
%
\begin{align}
\label{eq:intervaldistance}
\intd(v, v_{min}, v_{max}) = %| ( |v_{min}| - |v | ) | + | (|v_{max}| - |v | )| 
	\begin{cases}
    		\hfil v_{min} - v & \text{if } v <  v_{min} \\
    		\hfil 0 & \text{if } v_{min} \leq v \leq v_{max} \\
    		\hfil v - v_{max}  & \text{if } v > v_{max} \
  	\end{cases}  	 
\end{align}
%
\noindent
where $\intd$ is a function that measures the distance of a value from an interval of desired ones (see figure \ref{fig:interval}). Equation \ref{eq:lowriskres} is used for $H_e$. Equation \ref{eq:lowriskdegh} promotes rooms with a low-medium number of passages. Both are normalized in the range $[0,1]$. We empirically set the weights as $w_D = 1 $ and $ w_{H_e} = 0.5 $.

\begin{figure}
\centering
\begin{minipage}[t]{0.4825\linewidth}
\includegraphics[width=\linewidth]{interval}
\caption{The values assumed by $\intd$ with $[0.3, 0.5]$ set as interval of desired values.}
\label{fig:interval}
\end{minipage}
\hfill
\begin{minipage}[t]{0.4825\linewidth}
\includegraphics[width=\linewidth]{visibility_medium}
\caption[How the \<visibility heuristic> defined for health packs varies depending on the tile node degree.]{How the \<visibility heuristic> defined for health packs varies depending on the tile node degree, with the degree ranging in $[0, 15]$.}
\label{fig:visibility_medium}
\end{minipage}
\end{figure}

% TILE %

\paragraph{Tile selection}

Considering equation \ref{eq:tile_heuristic}  and given the visibility graph of the map $G_v$, the subset of tile nodes that belong to the room ($T \subset G_v$) and the subset of tile nodes of the room which contain a game element ($H \subset T$), the most suitable tile for containing a health pack is selected using the following heuristic:
%
\begin{align}
\label{eq:lowvish}
v(t) = 1 - \biggg| 0.5 - \cfrac{\degcent(t) - \min_{t' \in G_v}\degcent(t')}{\max_{t' \in G_v}\degcent(t') - \min_{t' \in G_v}\degcent(t')} \biggg| 
\end{align}
%
\noindent
that promotes tiles with medium visibility (see figure \ref{fig:visibility_medium}). Equations \ref{eq:lowwallclos} and \ref{eq:lowresclos} are used for $h_w(t)$ and $h_e(t)$, respectively. All three are normalized in the range $[0,1]$. We empirically set the weighs to $ w_v = 1 $, $ w_{h_w} = 0.25 $ and $ w_{h_e}  = 0.5 $.

% AMMUNITION %

\subsubsection{Ammunition}

For ammunition, the heuristic-based approaches for rooms and tiles respectively select rooms that have either a low-medium or a high number of connections and are distant from each other and tiles that offer the best balance between high visibility and distance from the walls. In this way ammunition is placed in areas that are either easy or difficult to reach and easy to spot.

% ROOM %

\paragraph{Room selection}

Considering equation \ref{eq:room_heuristic}  and given the rooms and corridors graph of the map ($G_{rr}$), the subset of room nodes ($R \subseteq G_{rr}$) and the subset of element nodes ($S \subset G_{rr}$), the most suitable room for containing ammunition is selected using equations \ref{eq:lowriskres} and \ref{eq:lowriskdegh}, with
%
\begin{align}
\label{eq:degfita1}
f_{deg}(r) = \intd(\degcent(r),0.2,0.4)
\end{align}
%
\noindent
for obtaining rooms with a low-medium number of connections and
 %
\begin{align}
\label{eq:degfita2}
f_{deg}(r) = \intd(\degcent(r),0.8,0.9)
\end{align}
%
\noindent
for obtaining rooms with a high number of connections. We empirically set the weighs to $w_D = 1 $ and $ w_{H_e} = 0.25 $.

% TILE %

\paragraph{Tile selection}

Considering equation \ref{eq:tile_heuristic} and given the visibility graph ($G_v$), the subset of tile nodes that belong to the room ($T \subset G_v$) and the subset of tile nodes of the room which contain a game element ($H \subset T$), the most suitable tile for containing ammunition is selected using the following heuristic:
%
\begin{align}
\label{eq:highvisa}
v(t) = \cfrac{\degcent(t) - \min_{t' \in T}\degcent(t')}{\max_{t' \in T}\degcent(t') - \min_{t' \in T}\degcent(t')}
\end{align}
%
\noindent
that promotes tiles with high visibility. Equations \ref{eq:lowwallclos} and \ref{eq:lowresclos} are used for $h_w(t)$ and $h_e(t)$, respectively. All three are normalized in the range $[0,1]$. We empirically set the weights as $ w_v = 1 $, $ w_{h_w} = 0.25 $ and $ w_{h_e}  = 0.5 $.

% WEAPONS %

\subsection{Weapon placement}

For what concerns the weapons provided by the framework we have not defined any specific heuristic, but, beside the assault rifle which is always available to the player, they could be placed as follows:

\begin{itemize}
\item \<Shotgun>: since it is a medium damage weapon it could be positioned in a room that has a medium number of connections and is relatively close to a spawn point. 
\item \<Rocket launcher>: since it is a high damage weapon it could be positioned in a room that has a high number of connections, creating an interesting collision point, or in a dead end, where its utility is limited.
\item \<Sniper rifle>: since it is a high damage weapon it could be positioned in a room that has a high number of connections, creating an interesting collision point.
\end{itemize}

% SUMMARY %

\section{Summary}

In this chapter we analyzed the approach that we have defined to perform map analysis and placement of game elements using Graph Theory. After an overview of the graphs and metrics that allow to highlight interesting information about a map, we listed the rules and patterns commonly used to position game elements in deathmatch maps and we described how we converted them into heuristics.

% CHAPTER 5 - FIRST EXPERIMENT %

\chapter{A case study: spawn points placement}

% INTRODUCTION %

In this chapter we discuss a case study we designed to test the effectiveness of our framework. We
performed an experiment with real users to validate the placement heuristics for spawn points and, at the same time, to test the data-collection capabilities of our framework, that was used to setup and manage the experiment.

% DESCRIPTION %

\section{Goals}

We designed this experiment to analyze how our methods for the placement of spawn points influence the up-player vs down-player dynamic and to prove their effectiveness. We tried to recreate the situation where the up-player, once killed his opponent, tries to find him as soon as possible, just after his respawn, to score another easy kill. As we have seen, a well-designed map should slow down this operation by having its spawn points in areas that are not central, are easy to leave and covered. In this experiment we compared two approaches: the \<uniform> one, that selects rooms with the uniform method (see subsubsection \ref{sss:room_selection}) and tiles with the heuristic defined for spawn points, and the \<heuristic> one, that selects both rooms and tiles with the heuristics defined for spawn points (see subsubsection \ref{sss:sph}) and should allow to obtain a placement of spawn point coherent with the one described above. These two approaches use the same method to select tiles in order to focus on the effects of room selection, since the ones of tile selection are rather obvious (an object on a tile with a low visibility is difficult to find).

\par

To highlight this gameplay dynamic, we designed a game mode where the user, which represents the up-player, must find and destroy a static target, which represents the down-player, as many times as possible before times runs out. Each time that the user destroys a target, it respawns at a random spawn point. The user cannot die and has infinite ammunition, so he does not have to look for resources.

% SETUP %

\section{Experimental design}

For this experiment, we setup the \<Experiment Manager> to propose in each play session a quick tutorial, two matches and a survey. The experiment was composed by three \<studies>, corresponding to three different maps, each one composed by two \<cases>, one corresponding to the map populated with the heuristic approach and one corresponding to a pool of five versions of the map populated with the uniform approach\footnote{When selecting rooms with the uniform method the first one is chosen at random. This allows to obtain multiple versions of the same map.}. In a play session, the user played the same map twice, once with the heuristic distribution and once with the uniform distribution, in a random order and with the map flipped in one of the two matches. We used the survey to profile each user according to their skill and familiarity with video games and FPS and to get a feedback about the match in which they found it harder to locate the targets. The experiment was deployed online and played by the users via browser on their own computers.

\par

As game mode, we used \<Target Hunt>, with the game duration set to three minutes, the list of spawnable entities composed of just one target and an \<assault rifle> with infinite ammunition as the only weapon available to the player. The maps were stored as text files and displayed with the \<Prefab Assembler>. 

\par

For each match, a complete game log was saved, along with the following performance metrics, saved in a separate log:
\begin{itemize}
\item \<TargetLogs>: this field contains a list of all the targets that the user managed to destroy. Each entry contains a timestamp, the coordinates of the destroyed target, the coordinates of the user, the distance covered by the user and the time passed during the lifespan of the target.
\item \<Shots>: the total number of projectiles shot by the user.
\item \<Hits>: the number of projectiles that hit a target.
\item \<Accuracy>: the percentage of projectiles that hit a target.
\item \<Kills>: the total number of targets destroyed by the user.
\item \<Distance>: the total distance covered by the user during the match, considering his complete trajectory and cells of unitary width as reference unit.
\item \<AvgKillTime>: the average time needed for the user to find a target, computed as the duration of the match divided by the number of kills.
\item \<AvgKillDistance>: the average distance covered by the user to find a target, computed as \<Distance> divided by the number of kills.
\end{itemize}

\noindent
The performance of the player is measured by \<AvgKillTime> and \<AvgKillDistance>, that are also indicators of how difficult it is to find targets in the map. The answers to the survey were saved as well.

\par

The three maps that we have selected for this experiment have very different layouts:

\begin{itemize}
\item \<Arena>: this map presents a wide arena, two sides of which are adjacent to parallel corridors with many openings. As the visibility heatmap in figure \ref{img:arena_visibility} shows, the central arena allows to control most of the map, whereas the corridors offer some repair and perfect spots to place spawn points. Figure \ref{img:arena_safe} shows the spawn points positioned using the heuristic approach, whereas figure \ref{img:arena_uniform} shows one of the five configuration produced with the uniform approach.
\item \<Corridors>: this map presents many small rooms connected by long corridors. As it can be seen in figure \ref{img:corridors_visibility}, there is no area that allows to control the others and the only points with high visibility are the ones where corridors intersect. Figure \ref{img:corridors_safe} shows the spawn points positioned using the heuristic approach, whereas figure \ref{img:corridors_uniform} shows one of the five configuration produced using the uniform approach.
\item \<Intense>: compared to the previous two, this map presents an intermediate layout, since it has both open areas and small rooms connected by corridors. As it can be seen in figure \ref{img:intense_visibility}, this reflects also on the visibility, that is high in the open areas and low in the remaining sections of the map. Figure \ref{img:intense_safe} shows the spawn points positioned using the heuristic approach, whereas figure \ref{img:intense_uniform} shows one of the five configuration produced using the uniform approach.
\end{itemize}

\begin{figure}[p]
\centering
\begin{subfigure}[t]{0.3\linewidth}
\includegraphics[width=\linewidth]{arena_visibility}
\caption{Heatmap showing the visibility of the level.}
\label{img:arena_visibility}
\end{subfigure} 
\hfil
\begin{subfigure}[t]{0.3\linewidth}
\includegraphics[width=\linewidth]{arena_safe}
\caption{Spawn points (in red) placed using the heuristic approach.}
\label{img:arena_safe}
\end{subfigure}
\hfil
\begin{subfigure}[t]{0.3\linewidth}
\includegraphics[width=\linewidth]{arena_uniform}
\caption{Spawn points (in red) placed using the uniform approach.}
\label{img:arena_uniform}
\end{subfigure} 
\caption{``Arena'' map used in the experiment.}
\label{img:arena}
\vspace{0.4cm}
\begin{subfigure}[t]{0.3\linewidth}
\includegraphics[width=\linewidth]{corridors_visibility}
\caption{Heatmap showing the visibility of the level.}
\label{img:corridors_visibility}
\end{subfigure} 
\hfil
\begin{subfigure}[t]{0.3\linewidth}
\includegraphics[width=\linewidth]{corridors_safe}
\caption{Spawn points (in red) placed using the heuristic approach.}
\label{img:corridors_safe}
\end{subfigure}
\hfil
\begin{subfigure}[t]{0.3\linewidth}
\includegraphics[width=\linewidth]{corridors_uniform}
\caption{Spawn points (in red) placed using the uniform approach.}
\label{img:corridors_uniform}
\end{subfigure} 
\caption{``Corridors'' map used in the experiment.}
\label{img:corridors}
\vspace{0.4cm}
\begin{subfigure}[t]{0.3\linewidth}
\includegraphics[width=\linewidth]{intense_visibility}
\caption{Heatmap showing the visibility of the level.}
\label{img:intense_visibility}
\end{subfigure} 
\hfil
\begin{subfigure}[t]{0.3\linewidth}
\includegraphics[width=\linewidth]{intense_safe}
\caption{Spawn points (in red) placed using the heuristic approach.}
\label{img:intense_safe}
\end{subfigure}
\hfil
\begin{subfigure}[t]{0.3\linewidth}
\includegraphics[width=\linewidth]{intense_uniform}
\caption{Spawn points (in red) placed using the uniform approach.}
\label{img:intense_uniform}
\end{subfigure}
\caption{``Intense'' map used in the experiment.}
\label{img:intense} 
\end{figure}

As we have said, the heuristic and the uniform approaches select rooms with two different criteria, but they employ the same logic when selecting tiles. This means that the two heuristics select tiles which have similar visibility conditions, so the player's performance depends exclusively on how the rooms have been selected. This is observable in figures \ref{img:arena}, \ref{img:corridors} and \ref{img:intense}: when the uniform approach happens to select a room that has been selected also by the heuristic one, the spawn point is placed exactly on the same tile.

% RESULTS %

\section{Results}

\begin{table}
\setlength\extrarowheight{2pt}
\begin{tabularx}{\textwidth}{|l|C|C|C|C|C|}
\cline{3-6}
\multicolumn{2}{c|}{} & Total & Arena & Corridors & Intense \\
\hline
\multicolumn{2}{|l|}{Number of samples} & 27 & 10 & 9 & 8 \\
\hline
\multirow{ 2}{*}{$\mu(\text{Shots})$} & Heuristic & 104.04 & 114.89 & 101.00 & 94.88 \\
\cline{2-6}
& Uniform & 117.44 & 131.33 & 100.25 & 119.00 \\
\hline
\multirow{ 2}{*}{$\mu(\text{Hits})$} & Heuristic & 42.68 & 48.44 & 43.50 & 35.38 \\
\cline{2-6}
& Uniform & 50.12 & 54.22 & 44.50 & 51.13 \\
\hline
\multirow{ 2}{*}{$\mu(\text{Accuracy})$} & Heuristic & 0.45 & 0.43 & 0.51 & 0.40 \\
\cline{2-6}
& Uniform & 0.48 & 0.45 & 0.52 & 0.48 \\
\hline
\multirow{ 2}{*}{$\mu(\text{Kills})$} & Heuristic & 10.06 & 10.75 & 10.44 & 8.75 \\
\cline{2-6}
& Uniform & 11.88 & 12.27 & 10.67 & 12.75 \\
\hline
\multirow{ 2}{*}{$\mu(\text{Distance})$} & Heuristic & 675.22 & 633.56 & 675.43 & 716.68 \\
\cline{2-6}
& Uniform & 680.79 & 649.62 & 688.34 & 704.42 \\
\hline
\multirow{ 2}{*}{$\mu(\text{AvgKillTime})$} & Heuristic & 20.24 & 21.09 & 18.25 & 21.40  \\
\cline{2-6}
& Uniform & 16.06 & 15.50 & 17.46 & 15.18 \\
\hline
\multirow{ 2}{*}{$\mu(\text{AvgKillDistance})$} & Heuristic & 73.34 & 64.21 & 70.90 & 84.90 \\
\cline{2-6}
& Uniform & 60.70 & 55.24 & 68.13 & 58.72 \\
\hline
\multirow{ 2}{*}{$\mode(\text{Difficulty})$} & Effective & heuristic & heuristic & uniform & heuristic \\
\cline{2-6}
& Percived & heuristic & heuristic & equal & heuristic \\
\hline
\end{tabularx}
\caption[The information retrieved from the dataset.]{The information retrieved from the dataset. $\mu$ denotes the mean value and $\mode$ denotes the modal value.}
\label{tab:means}
\end{table}

The data collected with this experiment consisted of 27 samples. As table \ref{tab:means} shows, of these 27 samples, 10 are pairs of matches played in map ``Arena'', 9 are pairs of matches played in map ``Corridors'' and 8 are pairs of matches played in map ``Intense''. The table also shows the values of the metrics that we have defined classified by map and placement method.

\par

As we have seen, the performance of the player is measured with \<AvgKillTime> and \<AvgKillDistance> and by computing their mean values we observed that the users performed better in the matches associated to the uniform approach with respect to the ones associated to heuristic approach. For the former the mean value of \<AvgKillTime> is $16.06$ seconds and the one of \<AvgKillDistance> is $60.7$ cells, whereas for the latter the mean value of \<AvgKillTime> is $20.24$ seconds and the one of \<AvgKillDistance> is $73.34$ cells. The respective increase of $26\%$ and $21\%$ on the average time and distance needed to find a target confirms that spawn points placed with the heuristic approach are more difficult to find than the ones placed with the uniform approach.

\par

To test the statistical significance of this result, we performed the \<Wilcoxon> statistical test\footnote{The \<Wilcoxon signed-rank test> is a \<non-parametric statistical hypothesis test> used to compare two matched samples to assess whether their population mean ranks differ.} by \<Pratt>\footnote{With respect to the standard Wilcoxon test, the one by Pratt considers also the observations for which the difference of the elements in the pair is zero. We opted for this approach since some samples happen to have metrics with the same value for the two placements.}, using as matched samples the values assumed by the metric at issue when using heuristic placement and when using uniform placement. Both \<AvgKillTime> and  \<AvgKillDistance> passed the test, the former with $\alpha = 0.00203 < 0.005$, one-tiled, and the latter with $\alpha = 0.01243 < 0.05$, one-tiled.

\begin{figure}
\centering
\begin{subfigure}[t]{0.49\linewidth}
\includegraphics[width=\linewidth]{scatter_kills}
\caption{Kills.}
\label{img:scatter_kills}
\end{subfigure}
\hfill
\begin{subfigure}[t]{0.49\linewidth}
\includegraphics[width=\linewidth]{scatter_distance}
\caption{Distance.}
\label{img:scatter_distance}
\end{subfigure}

\begin{subfigure}[t]{0.49\linewidth}
\includegraphics[width=\linewidth]{scatter_shots}
\caption{Shots.}
\label{img:scatter_shots}
\end{subfigure}
\hfill
\begin{subfigure}[t]{0.49\linewidth}
\includegraphics[width=\linewidth]{scatter_accuracy}
\caption{Accuracy.}
\label{img:scatter_accuracy}
\end{subfigure}
\caption[Experiment outcomes by metrics.]{Experiment outcomes by metrics. The outcome associated to the heuristic approach is on the horizontal axis, the one associated to the uniform approach is on the vertical axis.}
\label{img:metrics} 
\end{figure}

The effect of the two approaches on the metrics can also be analyzed by plotting their values, assigning to the horizontal axis the value of the metric in maps with heuristic placement and to the vertical axis the value of the metric in maps with uniform placement. Each point of such graph represents the outcome of a test whose coordinates are the values of the metric in the two matches. By tracing the bisector, it is easy to see for which of the two approaches a metric is higher. If the points are scattered under the bisector it means that the metric tends to be higher for the heuristic approach, whereas if they are scattered above the bisector it means that the metric tends to be higher for the uniform approach. Figure \ref{img:metrics} shows such graphic for each metric:

\begin{itemize}
\item \<Kills>: as figure \ref{img:scatter_kills} shows, the number of kills tends to be higher with the uniform placement ($11.88 > 10.06$, considering the mean values). This result is rather obvious, since, as we have seen analyzing \<AvgKillTime> and \<AvgKillDistance>, targets are more difficult to find with the heuristic placement. The Wilcoxon test, which is passed with $\alpha = 0.00347 < 0.005$, one-tiled, confirms this outcome. 
\item \<Distance>: as figure \ref{img:scatter_distance} shows, the total distance covered by users is not influenced by the employed placement method ($680.79\approx675.22$, considering the mean values). This is due to the fact that users are constantly moving and the agility with which they navigate the map depends exclusively on their familiarity with FPS games and on the layout of the map. The Wilcoxon test, which is not passed with $\alpha = 0.294 > 0.05$, one-tiled, confirms this outcome.
\item \<Shots>: as figure \ref{img:scatter_shots} shows, the number of shots tends to be higher with the uniform placement ($117.44 >  104.04$, considering the mean values). This result is coherent with what we have observed for \<Kills>, to which \<Shots> is expected to be directly proportional. The Wilcoxon test, which is passed with $\alpha = 0.0336 < 0.05$, one-tiled, confirms this outcome.
\item \<Accuracy>: as figure \ref{img:scatter_accuracy} shows, the accuracy is not significantly influenced by the used placement method ($45\% \approx 48\%$, considering the mean values). This is due to the fact that the accuracy depends almost exclusively on the aiming skills of the user. The Wilcoxon test, which is not passed with $\alpha = 0.294 > 0.05$, one-tiled, confirms this outcome.
\end{itemize}

We also observed that the effects of the placement were considerably different depending on the layout of the map. With the heuristic placement, map ``Arena'' had a mean number of kills of $10.75$, map ``Corridors'' of $10.44$ and map ``Intense'' of $8.75$. We expected the one of ``Arena'' to be the highest, because of its central area that dominates the rest of the map, but we did not expect the ones of ``Corridors'' to be almost as high, since its structure is more difficult to navigate. Moreover, we expected an intermediate number of kills in ``Intense'', since it merges the features of the two other maps, but it proved to be the map where targets were harder to find. This could be explained by the fact that ``Corridors'', even if more complex than ``Arena'', has a rather regular structure and its long corridors allow to easily spot a target. Instead, for what concerns ``Intense'', its structure is rather complex and difficult to navigate and since the central open area does not allow to control all of the map the tactical advantage it provides is not so strong. With the uniform placement the mean number of kills of ``Arena'' increased to $11.88$, the one of ``Corridors'' remained almost the same ($10.67$) and the one of ``Intense'' increased dramatically to $12.75$. The reasons for such different reactions lies in the layout of each map. The central area of ``Arena'' allows to control all of the surroundings, so the placement of spawn points in areas that are less visible has a relevant effect. ``Corridors'', instead, has a regular structure and the number of intersections between its corridors is almost always the same, so it is not relevant where spawn points are placed, since all the rooms have the same features. Finally, the tangled structure of ``Intense'' offers to the heuristic approach a lot of interesting spots where to place spawn points and the presence of an area that allows a partial control of the map makes this choice even more meaningful. This shows that the effects of the heuristic approach are more pronounced in maps which layout is not uniform.

\begin{figure}
	\centering
	\hfill
  	\begin{subfigure}[t]{0.385\linewidth}
		\includegraphics[width=\linewidth]{heat_arena_safe}
     		\caption{Map ``Arena'' with heuristic placement.}
     		\label{img:heat_arena_safe}
 	\end{subfigure}
 	\hfill
  	\begin{subfigure}[t]{0.385\linewidth}
    		\includegraphics[width=\linewidth]{heat_arena_uniform}
     		\caption{Map ``Arena'' with uniform placement.}
     		\label{img:heat_arena_uniform}
  	\end{subfigure}
  	\hfill
  	
  	\hfill
  	\begin{subfigure}[t]{0.385\linewidth}
		\includegraphics[width=\linewidth]{heat_corridors_safe}
     		\caption{Map ``Corridors'' with heuristic placement.}
     		\label{img:heat_corridors_safe}
 	\end{subfigure}
 	\hfill
  	\begin{subfigure}[t]{0.385\linewidth}
    		\includegraphics[width=\linewidth]{heat_corridors_uniform}
     		\caption{Map ``Corridors'' with uniform placement.}
     		\label{img:heat_corridors_uniform}
  	\end{subfigure}
 	\hfill
 	
 	\hfill
  	\begin{subfigure}[t]{0.385\linewidth}
		\includegraphics[width=\linewidth]{heat_intense_heuristic}
     		\caption{Map ``Intense'' with heuristic placement.}
     		\label{img:heat_intense_heuristic}
 	\end{subfigure}
 	\hfill
  	\begin{subfigure}[t]{0.385\linewidth}
    		\includegraphics[width=\linewidth]{heat_intense_uniform}
     		\caption{Map ``Intense'' with uniform placement.}
     		\label{img:heat_intense_uniform}
  	\end{subfigure}	
  	\hfill
	\caption[Heat maps of the player position for the three maps used in the experiment.]{Heat maps of the player positions for the three maps used in the experiment. The red circles represent spawn points.}
	\label{img:heat}
\end{figure}

The position of spawn points also influenced the way in which users moved across the maps. Figure \ref{img:heat} shows, for each map and for each placement method, the heat maps of the sections that where crossed more frequently by the users. It is possible to notice that users, once understood the topology of the map, started to follow well defined \<farming routes>\footnote{In video games, \<farming routes> are regular closed paths defined to maximize the collection of certain resources in a specific map.}. These routes tend to be circular and to skirt the perimeter of the maps, with deviations that are influenced by the position of spawn points. The overall circularity is remarked by the heat maps associated to the uniform positioning, that being obtained by a set of five different uniform placements\footnote{The uniform \<study> has five \<cases>, whereas the heuristic \<study> has just one.}, represent an average route with respect to the possible position of spawn points.

\begin{figure}
\centering
\includegraphics[width=0.8\linewidth]{difficulty}
\caption[Comparison between the effective and the perceived difficulty.]{Comparison between the effective and the perceived difficulty. Each square contains the number of samples where the user performed worst in the match with placement \<Effective> and evaluated as more difficult the match with placement \<Percived>. The match where a user performed worst is the one with the highest \<AvgKillTime>.}
\label{img:percived} 
\end{figure}

It is also interesting to notice how the difficulty in finding the targets was perceived by the users. As it can be seen in figure \ref{img:percived}, in the survey the majority of the users classified as more difficult the match where they performed worst, but some of them answered with the one where they performed better. 

\par

Finally, the intuition of flipping one of the two maps was correct, since many users believed to have played two completely different maps. 

% SUMMARY %

\section{Summary}

In this chapter we analyzed the experiment that we setup and its results, that proved the effectiveness of our heuristics and of our framework. We observed that the heuristic placement makes the targets harder to find, whereas the total distance covered by the user and the shooting accuracy are not influenced by the employed placement method. We also discovered that our heuristics work better with maps that do not have an uniform topology and we observed that users tend to define and follow specific routes when performing a research.

% CHAPTER 7 - CONCLUSIONS %

\chapter{Conclusions}

% CONCLUSIONS %

The purpose of this thesis was to create a framework to perform research in Procedural Content Generation for First Person Shooters and to attempt a new approach to level design analysis and game element placement. 

\par

Past works have employed open source games, like Cube 2, that allow to perform validation via artificial agents but present many limitations when it is needed to collect information from real users. There was therefore the need of a way to collect data online, in an easy and quick way, and we answered to it by designing a framework to deploy browser playable experiments, that once defined collect data automatically. Being aimed at research, we wanted our framework to be as versatile as possible, so we opted for a modular and parametric design that is easy to customize and we included many generation algorithms and map representation formats, both single-level and multi-level, that have been used in previous works. We included the All Black format, defined by Cardamone et al.\cite{Cardamone:2011:EIM:2008402.2008411}, that is a standard in the literature, but we extended it to be more complete and flexible, introducing variable genome size, game elements codification and multi-level support.

\par

We explored how Graph Theory can be applied to level design, with regard to both map analysis and placement of game elements. For the former, we defined various graphs that can be generated from the All-Black representation of a map, each one highligthing different features such as the visibility or the reachability of tiles and rooms, and we selected some indicators from Graph Theory that allow to obtain topological information about the map at issue. For the latter, we defined an approach that uses heuristics to place game elements, taking into account their specific features and the indicators that we have selected. To define these heuristic, we analyzed how game elements influence the up-player vs down-player dynamic and how they should be positioned to create an engaging and balanced gameplay. This new approach to the subject proved to be very interesting, since it allows to analyze level design from a new perspective and to easily define topological rules.

\par

Finally, we tested our framework by performing an experiment to analyze how the placement of spawn points influences the up-player vs down-player dynamic. With this experiment we were able to validate the placement heuristics that we have defined and we managed to observe how the map layout influences the disposition of game elements. In this way, we proved our graph-based approach to be useful both for map analysis and for the contextual positioning of game elements.

% ISSUES %

\section{Known issues and possible criticism}

The main issue with the framework is that it does not have neither artificial agents nor the support for online multiplayer and this limits its possible applications.

\par

For what concerns graph analysis, the rules that we have defined for placing game elements could be criticized for a lack of a strong theoretical basis, since as we have seen there is still no common ground for what concerns level design. Moreover, we assigned the weights used in the placement heuristics empirically, making various attempts and choosing the weights that produced the disposition of game elements most coherent with the rules we defined. Despite this, the experiment proved both the rules and the weight assignment to be effective.

% FUTURE %

\section{Future developments}

Two major features that should be implemented in the framework are an artificial intelligence for agents and the support for online multiplayer, since they would allow to significantly increase the possible applications of our work. Moreover, to make the framework more complete and allow to directly generate well designed maps, it would be a great improvement to implement the map analysis and the game element placement directly in the framework, instead of performing them using an external tool.

\par

In subsection \ref{ss:interesting_metrics} we listed many metrics that can give interesting information about the layout of a map, but we have used only some of them to define the placement heuristics. An interesting development would be to include more of them, in particular the ones that allow to define areas of the map, like \<Periphery> and \<Center>. As we have highlighted, weapons require a specific treatment when positioned, since their overall damage, strengths and weakness should influence their place in the map, and such metrics can be employed to define the areas that better suit each weapon. These new heuristics, as well as the already defined ones, would benefit of an experimental analysis similar to the one used to validate the heuristics for the placement of spawn points. Another improvement would be the extension of the analysis performed via graph to multi-level maps. Moreover, as we have already highlighted in the thesis, the graphs we have defined could be used for the individuation and analysis of design patterns.

\par

Finally, it would be interesting to design an evolutionary process that generates maps and places resources using a fitness function addressed to the up-player vs down-player dynamic.

% BACK MATTER %

\backmatter

% BIBLIOGRAPHY %

\bibliographystyle{ieeepes}
\bibliography{bibliography} 

\end{document}