\chapter{Introduction}

Video games are really complex products, but it is rather simple to point out three elements that mainly influence their commercial success: \<visuals>, \[gameplay]\footnote{The specific way in which players interact with the game.} and \<narrative>. If in the past visual improvements were impressively fast, with games looking more and more realistic from year to year, lately this progress consistently slowed down, leaving narrative and gameplay as the main selling points. Since video games are interactive products, the latter is the one that influences user experience the most. Gameplay is defined by a set of rules commonly referred as \<game design>. To the player, game design is presented in a tangible way via \<level design>, which consists in the creation of the worlds where the game takes place. This is a critical component, since an inadequate level design can easily compromise the whole experience.

\par

One of the most successful video game genre is the one of \[First Person Shooters], that thanks to its first player perspective allows the user to experiment a complete immersion in the game world. From the very beginning, this has required a close attention to level design, that underwent a constant evolution, up to the stable situation of the recent years. Furthermore, the ever-increasing success of competitive multiplayer added a new level of complexity to the creation of maps, that need to support different game modes, play styles and interactions, allowing a fun and challenging gameplay to arise naturally.

\section{Motivations and purpose}

Despite the importance of level design for the FPS genre, the video game industry has never attempted a scientific approach to this field. Consequently, game design is a rather abstract discipline, with no common vocabulary or well-defined standards, but rather based on the experience of who is working in this field from many years. This affects also the related literature, that is confined to listing the most used patterns and conventions, without focusing on why they work.

\par

In the last years, instead, academic environments started to address this discipline with increasing interest. The researches performed in this field revolve around the identification and definition of design patterns and design techniques, with a deep analysis of how and why they work, and the creation of novel approaches to level design. Some of these techniques try to automate the design process by employing \<procedural generation>, often combined with \<evolutionary algorithms>. 

\par

A problem peculiar to this kind of research is how difficult it is to obtain information from real users, since this requires them to either download a specific game or to take part in play-test sessions and both option discourage participation.

\par

The aim of this thesis is to solve this problem and to attempt a novel approach to map analysis and game element placement. The former was addressed by designing an open-source \<framework> capable of generating and importing single-level and multi-level maps and of deploying online experiments to collect data from real users via a browser-playable FPS game. The latter, instead, was achieved by using \<Graph Theory> to compute topological metrics from multiple graph representation of the maps, each one highlighting different features, and by using the information provided by these metrics to strategically place game elements, paying attention to the gameplay dynamics associated with each one of them.

\section{Synopsis}

The contents of the thesis are the followings:

\par \mbox{}

In the second chapter we describe the state of the art of First Person Shooters, both in academic research and in commercial games, paying attention on how level design practices and procedural content generation are applied in this genre. We also list some examples of how Graph Theory is employed in video games.

\par
\mbox{}

In the third chapter we present the framework that we have developed, analyzing its features and its overall structure.

\par
\mbox{}

In the fourth chapter we present an approach that uses Graph Theory to place game elements in procedurally generated levels, with an overview of the theory and the assumptions behind it.

\par
\mbox{}

In the fifth chapter we describe an experiment performed with our framework for validating the heuristics that we have defined for the placement of spawn points.

\par 
\mbox{}

In the concluding chapter we evaluate the obtained results and we analyze the potential future developments of this work.